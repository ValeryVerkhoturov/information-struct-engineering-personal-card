\documentclass[10pt, a4paper, titlepage]{article}

\usepackage[T2A]{fontenc}
\usepackage[utf8]{inputenc}
\usepackage[russian]{babel}

\usepackage{hyperref}
\hypersetup{pdftitle={Методология структур данных. Задание 2}, pdfauthor={В. С. Верхотуров}, colorlinks=false, pdfborder={0 0 0}}


\usepackage{array}
\usepackage{longtable}
\usepackage{xtab}
\setlength{\columnsep}{5mm}

\usepackage{authblk}

% Titlepage

\title{Методология проектирования структур данных. Задание 2}
\author{В. С. Верхотуров}
\affil{БСБО-05-20}
\affil{РТУ МИРЭА}
\date\today


% Macros

%% Keys

%%% Primary Key
\newcommand{\pk}[1]{\textbf{#1}}

%%% Foreign Key
\newcommand{\fk}[1]{\textit{#1}}

%%% Primary Foreign Key
\newcommand{\pfk}[1]{\pk{\fk{#1}}}

%% Columns
\newcommand{\firstColumn}[4]{#1 --- \newline #2 --- \newline #3 \newline\newline #4}

\newcommand{\thirdColumn}[6]{
#1 \newline 
\underline{Первичный ключ} --- #2 \newline 
\setbox0=\hbox{#3\unskip}\ifdim\wd0=0pt
\else
  \underline{Внешний(е) ключ(-и)}: #3 \newline
\fi
#4 \newline 
\underline{Первичный ключ} --- #5 \newline
\setbox0=\hbox{#6\unskip}\ifdim\wd0=0pt
\else
  \underline{Внешний(е) ключ(-и)}: #6 \newline
\fi
}

\newcommand\generalizedColumn[6]{\thirdColumn{#1:}{#2}{#3}{#4:}{#5}{#6}}


%%  Rules

%%% 1
\newcommand\ruleOneMondatoryOneMondatoryNum{1}
\newcommand\ruleOneMondatoryOneMondatory{1 Об - 1 Об}

%%% 2
\newcommand\ruleOneMondatoryOneOptionalNum{2}
\newcommand\ruleOneMondatoryOneOptional{1 Об - 1 Н/О}
\newcommand\ruleOneOptionalOneMondatoryNum{2}
\newcommand\ruleOneOptionalOneMondatory{1 Н/О - 1 Об}

%%% 3
\newcommand\ruleOneOptionalOneOptionalNum{3}
\newcommand\ruleOneOptionalOneOptional{1 Н/О - 1 Н/О}

%%% 4
\newcommand\ruleOneOptionalManyMondatoryNum{4}
\newcommand\ruleOneOptionalManyMondatory{1 Н/О - М Об}
\newcommand\ruleManyMondatoryOneOptionalNum{4}
\newcommand\ruleManyMondatoryOneOptional{М Об - 1 Н/О}

\newcommand\ruleOneMondatoryManyMondatoryNum{4}
\newcommand\ruleOneMondatoryManyMondatory{1 Об - М Об}
\newcommand\ruleManyMondatoryOneMondatoryNum{4}
\newcommand\ruleManyMondatoryOneMondatory{М Об - 1 Об}

%%% 5
\newcommand\ruleOneMondatoryManyOptionalNum{5}
\newcommand\ruleOneMondatoryManyOptional{1 Об - М Н/О}
\newcommand\ruleManyOptionalOneMondatoryNum{5}
\newcommand\ruleManyOptionalOneMondatory{М Н/О - 1 Об}

\newcommand\ruleOneOptionalManyOptionalNum{5}
\newcommand\ruleOneOptionalManyOptional{1 Н/О - М Н/О}
\newcommand\ruleManyOptionalOneOptionalNum{5}
\newcommand\ruleManyOptionalOneOptional{М Н/О - 1 Н/О}

%%% 6
\newcommand\ruleManyOptionalManyOptionalNum{6}
\newcommand\ruleManyOptionalManyOptional{М Н/О - М Н/О}

%% Entities

%%% Работник
\newcommand\rabotnik{Работник (\pk{Табельный номер}, Фамилия, Имя, Отчество, Инициалы, ИНН, СНИЛС, Пол, Дата рождения, Место рождения, \fk{Код ОКАТО места рождения}, Номер паспорта, Дата выдачи паспорта, Орган выдачи паспорта, Адрес места жительства по паспорту, Индекс адреса места жительства по паспорту, Адрес места жительства фактический, Индекс адреса места жительства фактический, Дата регистрации по месту жительства, Номер телефона, Дополнительные сведения)}

\newcommand\rabotnikPK{Табельный номер}
\newcommand\rabotnikFK{Код ОКАТО места рождения}

%%% Старые ФИО
\newcommand\starieFIO{Старые ФИО (\pfk{Табельный номер}, \pk{Номер изменения фамилии}, ФИО, Документ - основание)}

\newcommand\starieFIOPK{Табельный номер, Номер изменения фамилии}
\newcommand\starieFIOFK{Табельный номер}

%%% Классификатор гражданства
\newcommand\kGrazhdanstva{Классификатор гражданства (\pk{Код гражданства}, Наименование гражданства)}

\newcommand\kGrazhdanstvaPK{Код гражданства}
\newcommand\kGrazhdanstvaFK{}

%%% Страны
\newcommand\strani{Страны (\pk{Код страны}, \pfk{Табельный номер}, Наименование страны, Дата выдачи, Вид документа, Наименование документа)}

\newcommand\straniPK{Табельный номер, Код страны}
\newcommand\straniFK{Табельный номер}

%%% Классификатор иностранного языка
\newcommand\kInostrannogoYazika{Классификатор иностранного языка (\pk{Код иностранного языка}, Наименование иностранного языка)}

\newcommand\kInostrannogoYazikaPK{Код иностранного языка}
\newcommand\kInostrannogoYazikaFK{}

%%% Классификатор степени знания иностранного языка
\newcommand\kStepeniZnaniaInostrannogoYazika{Классификатор степени знания иностранного языка (\pk{Код степени знания}, Наименование степени знания)}

\newcommand\kStepeniZnaniaInostrannogoYazikaPK{Код степени знания}
\newcommand\kStepeniZnaniaInostrannogoYazikaFK{}

%%% Знание иностранного языка
\newcommand\znanieInostrannogoYazika{Знание иностранного языка (\pfk{Табельный номер}, \pk{Код иностранного языка}, Код степени знания иностранного языка)}

\newcommand\znanieInostrannogoYazikaPK{Табельный номер, Код иностранного языка}
\newcommand\znanieInostrannogoYazikaFK{Табельный номер}

%%% Штатное расписание организации
\newcommand\shtatnoyeRaspisanieOrganizatsii{Организация штатного расписания (\pk{Номер штатного расписания}, Дата сопоставления, Период с, Период по, \fk{Номер приказа})}

\newcommand\shtatnoyeRaspisanieOrganizatsiiPK{Номер штатного расписания}
\newcommand\shtatnoyeRaspisanieOrganizatsiiFK{Номер приказа}

%%% Классификатор подразделения
\newcommand\kPodrazdeleniya{Классификатор подразделения (\pk{Код подразделения}, Наименование подразделения, Условное обозначение, Аббревиатура)}

\newcommand\kPodrazdeleniyaPK{Код подразделения}
\newcommand\kPodrazdeleniyaFK{}

%%% Классификатор должностей
\newcommand\kDolzhostey{Классификатор должностей (\pk{Код должности}, Наименование должности, Код ОКПДТР)}

\newcommand\kDolzhosteyPK{Код должности}
\newcommand\kDolzhosteyFK{}

%%% Штатные единицы
\newcommand\shtatnieEdinitsi{Штатные единицы (Кол - во штатных единиц, \pfk{Номер штатного расписания организации}, \pfk{Код подразделения}, \pfk{Код основной должности}, Тарифная ставка, Надбавка, \fk{Код другой должности}, \pk{Дата назначения}, Основание, \fk{Код основания прекращения трудового договора})}

\newcommand\shtatnieEdinitsiPK{Номер штатного расписания организации, Код подразделения, Код основной должности, Дата назначения}
\newcommand\shtatnieEdinitsiFK{Номер штатного расписания организации, Код подразделения, Код основной должности, Код другой должности, Код основания прекращения трудового договора}

%%% Приказ
\newcommand\prikaz{Приказ (\pk{Номер приказа}, Дата приказа, Тема)}

\newcommand\prikazPK{Номер приказа}
\newcommand\prikazFK{}

%%% Трудовой договор
\newcommand\trudovoiDogovor{Трудовой договор (\pk{Номер трудового договора}, Дата, Дата с, Дата по, Дата испытательного срока, Ставка, \fk{Код основания прекращения}, \fk{Код вида работы}, \fk{Код решения комиссии})}

\newcommand\trudovoiDogovorPK{Номер трудового договора}
\newcommand\trudovoiDogovorFK{Код основания прекращения, Код вида работы, Код решения комиссии}

%%% Классификатор характера работы
\newcommand\kHarakteraRaboti{Классификатор характера работы (\pk{Код характера}, Наименование характера)}

\newcommand\kHarakteraRabotiPK{Код характера}
\newcommand\kHarakteraRabotiFK{}

%%% Классификатор вида работы
\newcommand\kVidaRaboti{Классификатор вида работы (\pk{Код вида работы}, Наименование вида работы)}

\newcommand\kVidaRabotiPK{Код вида работы}
\newcommand\kVidaRabotiFK{}

%%% Классификатор основания прекращения трудового договора
\newcommand\kOsnovaniyaPrekrascheniaTrudovogoDogovora{Классификатор основания прекращения трудового договора (\pk{Код}, Наименование)}

\newcommand\kOsnovaniyaPrekrascheniaTrudovogoDogovoraPK{Код}
\newcommand\kOsnovaniyaPrekrascheniaTrudovogoDogovoraFK{}

%%% Классификатор образования
\newcommand\kObrazovania{Классификатор образования (\pk{Код образования}, Наименование образования)}

\newcommand\kObrazovaniaPK{Код образования}
\newcommand\kObrazovaniaFK{}

%%% Классификатор направления
\newcommand\kNapravlenia{Классификатор направления (\pk{Код ОКСО}, Наименование направления)}

\newcommand\kNapravleniaPK{Код ОКСО}
\newcommand\kNapravleniaFK{}

%%% Документ об образовании
\newcommand\documentObObrazovanii{Документ об образовании (\pfk{Табельный номер}, \pk{Номер}, Вид документа, Учреждение, Серия, Номер, Квалификация, \fk{Направление})}

\newcommand\documentObObrazovaniiPK{Табельный номер, Номер}
\newcommand\documentObObrazovaniiFK{Табельный номер, Направление}

%%% Стаж работы
\newcommand\stazhRaboti{Стаж работы (Дата с, Дата по, \pfk{Табельный номер}, \pk{Номер работы})}

\newcommand\stazhRabotiPK{Табельный номер, Номер работы}
\newcommand\stazhRabotiFK{Номер работы}

%%% Классификатор состояния в браке
\newcommand\kSostoyaniaVBrake{Классификатор состояния в браке (\pk{Код ОКИН}, Наименование)}

\newcommand\kSostoyaniaVBrakePK{Код ОКИН}
\newcommand\kSostoyaniaVBrakeFK{}

%%% Состояние в браке
\newcommand\sostoyaniyeVBrake{Состояние в браке (\pfk{Табельный номер}, \fk{Код ОКИН}, Дата с, Дата по, \pk{Номер изменения состояния})}

\newcommand\sostoyaniyeVBrakePK{Табельный номер, Номер изменения состояния}
\newcommand\sostoyaniyeVBrakeFK{Табельный номер, Код ОКИН}

%%% Классификатор родства
\newcommand\kRodstva{Классификатор родства (\pk{Код ОКИН}, Наименование)}

\newcommand\kRodstvaPK{Код ОКИН}
\newcommand\kRodstvaFK{}

%%% Родственники
\newcommand\rodtvenniki{Родственники (Фамилия, Имя, Отчество, \fk{Код родства}, Год рождения, \pfk{Табельный номер}, \pk{Номер родственника})}

\newcommand\rodtvennikiPK{Табельный номер, Номер родственника}
\newcommand\rodtvennikiFK{Код родства, Табельный номер}

%%% Воинский учет
\newcommand\voinsiyUchet{Воинский учет (\pfk{Табельный номер}, \fk{Код воинского звания}, \fk{Код состава (профиля)}, Полный код обозначения ВУС, Категория годности к военной службе, Наименование военного комиссариата по месту жительства, Общий воинский учет, Специальный воинский учет, Отметка о снятии воинского учета, \fk{Табельный номер работника кадровой службы}, Дата)}

\newcommand\voinsiyUchetPK{Табельный номер}
\newcommand\voinsiyUchetFK{Табельный номер, Код воинского звания, Код состава (профиля), Табельный номер работника кадровой службы}

%%% Классификатор воинских званий
\newcommand\kVoinskihZvaniy{Классификатор воинских званий (\pk{Код ОКИН}, Наименование)}

\newcommand\kVoinskihZvaniyPK{Код ОКИН}
\newcommand\kVoinskihZvaniyFK{}

%%% Классификатор составов
\newcommand\kSostavov{Классификатор составов (\pk{Код ОКИН}, Наименование)}

\newcommand\kSostavovPK{Код ОКИН}
\newcommand\kSostavovFK{}

%%% Аттестация
\newcommand\attestatsiya{Аттестация (Дата, \fk{Код решения комиссии}, Номер протокола, Дата протокола, Основание, \pfk{Табельный номер})}

\newcommand\attestatsiyaPK{Табельный номер}
\newcommand\attestatsiyaFK{Табельный номер, Код решения комиссии}

%%% Классификатор решений комиссий
\newcommand\kResheniyKomissii{Классификатор решений комиссий (\pk{Код решения комиссии}, Наименование)}

\newcommand\kResheniyKomissiiPK{Код решения комиссии}
\newcommand\kResheniyKomissiiFK{}

%%% Повышение квалификации
\newcommand\povishenieKvalifikatsii{Повышение квалификации (Дата с, Дата после, \fk{Код вида повышения квалификации}, Наименование, адрес, Наименование образовательного учреждения, Наименование документа, Дата документа, Серия, номер документа, Основание, \pfk{Табельный номер}, \pk{Номер})}

\newcommand\povishenieKvalifikatsiiPK{Табельный номер, Номер}
\newcommand\povishenieKvalifikatsiiFK{Код вида повышения квалификации, Табельный номер}

%%% Профессиональная переподготовка
\newcommand\professionalnayaPerepodgatovka{Профессиональная переподготовка (\pk{Дата с}, Дата по, Код специальности, Наименование документа, Номер документа, Дата, Основание, \pfk{Табельный номер})}

\newcommand\professionalnayaPerepodgatovkaPK{Дата с, Табельный номер}
\newcommand\professionalnayaPerepodgatovkaFK{Табельный номер}

%%% Награды, почетные звания
\newcommand\nagradiPochetnieZvaniya{Награды, почетные звания (\fk{Код награды}, Наименование документа, Номер документа, Дата документа, \pk{Номер награды}, \pfk{Табельный номер})}

\newcommand\nagradiPochetnieZvaniyaPK{Номер награды, Табельный номер}
\newcommand\nagradiPochetnieZvaniyaFK{Код награды, Табельный номер}

%%% Отпуск
\newcommand\otpusk{Отпуск (\fk{Код вида отпуска}, Период работы с, Периода работы по, Кол - во календарных дней отпуска, Дата с, Дата по, Основание, \pk{Номер отпуска}, \pfk{Табельный номер})}

\newcommand\otpuskPK{Номер отпуска, Табельный номер}
\newcommand{\otpuskFK}{Код вида отпуска, Табельный номер}

%%% Социальные льготы
\newcommand\sotsialnieLgoti{Социальные льготы (Наименование льготы, Номер документа, Дата документа, Основание, \pfk{Табельный номер}, \pk{Номер льготы}, \fk{Код вида отпуска})}

\newcommand\sotsialnieLgotiPK{Табельный номер, Номер льготы}
\newcommand\sotsialnieLgotiFK{Табельный номер, Код вида отпуска}

%%% Классификатор вида отпуска
\newcommand\kVidaOtpuska{Классификатор вида отпуска (\pk{Код}, Наименование)}

\newcommand\kVidaOtpuskaPK{Код}
\newcommand\kVidaOtpuskaFK{}

%%% Классификатор вида повышения квалификации
\newcommand\kVidaPovisheniyaKvalifikatsii{Классификатор вида повышения квалификации (\pk{Код}, Наименование)}

\newcommand\kVidaPovisheniyaKvalifikatsiiPK{Код}
\newcommand\kVidaPovisheniyaKvalifikatsiiFK{}

%%% Классификатор наград
\newcommand\kNagrad{Классификатор наград (\pk{Код}, Наименование)}

\newcommand\kNagradPK{Код}
\newcommand\kNagradFK{}


\begin{document}

\maketitle

\tableofcontents
\newpage

\twocolumn[
\section{Описание классификаторов}
\subsection{ОКПО}
Общероссийский классификатор предприятий и организаций --- классификатор, используемый для ведения списка юридических лиц во всех государственных классификаторах и базах данных РФ для облегчения связывания данных о юридических лицах и учета статистики.]
\begin{xtabular}{rp{.35\textwidth}}
    \hline
    \textbf{Код} & \textbf{Наименование} \\ \hline
    01 & Республика Адыгея (Адыгея) \\
    02 & Республика Башкортостан \\
    03 & Республика Бурятия \\
    04 & Республика Алтай \\
    05 & Республика Дагестан \\
    06 & Республика Ингушетия \\
    07 & Кабардино-Балкарская Республика \\
    08 & Республика Калмыкия \\
    09 & Карачаево-Черкесская Республика \\
    11 & Республика Коми \\
    12 & Республика Марий Эл \\
    13 & Республика Мордовия \\
    14 & Республика Саха (Якутия) \\
    15 & Республика Северная Осетия - Алания \\
    16 & Республика Татарстан (Татарстан) \\
    17 & Республика Тыва \\
    18 & Удмуртская Республика \\
    19 & Республика Хакасия \\
    20 & Чеченская Республика \\
    21 & Чувашская Республика – Чувашия \\
    22 & Алтайский край \\
    23 & Краснодарский край \\
    24 & Красноярский край \\
    25 & Приморский край \\
    26 & Ставропольский край \\
    27 & Хабаровский край \\
    28 & Амурская область \\
    29 & Архангельская область \\
    30 & Астраханская область \\
    31 & Белгородская область \\
    32 & Брянская область \\
    33 & Владимирская область \\
    34 & Волгоградская область \\
    35 & Вологодская область \\
    36 & Воронежская область \\
    37 & Ивановская область \\
    38 & Иркутская область \\
    39 & Калининградская область \\
    40 & Калужская область \\
    41 & Камчатская область \\
    42 & Кемеровская область \\
    43 & Кировская область \\
    44 & Костромская область \\
    46 & Курская область \\
    47 & Ленинградская область \\
    48 & Липецкая область \\
    49 & Магаданская область \\
    50 & Московская область \\
    51 & Мурманская область \\
    52 & Нижегородская область \\
    53 & Новгородская область \\
    54 & Новосибирская область \\
    55 & Омская область \\
    56 & Оренбургская область \\
    57 & Орловская область \\
    58 & Пензенская область \\
    59 & Пермская  \\
    60 & Псковская область \\
    61 & Ростовская область \\
    62 & Рязанская область \\
    63 & Самарская область \\
    64 & Саратовская область \\
    65 & Сахалинская область \\
    66 & Свердловская область \\
    67 & Смоленская область \\
    68 & Тамбовская область \\
    69 & Тверская область \\
    70 & Томская область \\
    71 & Тульская область \\
    72 & Тюменская область \\
    73 & Ульяновская область \\
    74 & Челябинская область \\
    75 & Читинская область \\
    76 & Ярославская область \\
    77 & г. Москва \\
    78 & г. Санкт-Петербург \\
    79 & Еврейская автономная область \\
    80 & Агинский Бурятский автономный округ \\
    81 & Коми-Пермяцкий автономный округ \\
    82 & Корякский автономный округ \\
    83 & Ненецкий автономный округ \\
    84 & Таймырский (Долгано-Ненецкий) автономный округ \\
    85 & Усть-Ордынский Бурятский автономный округ \\
    86 & Ханты-Мансийский автономный округ - Югра \\
    87 & Чукотский автономный округ \\
    88 & Эвенкийский автономный округ \\
    89 & Ямало-Ненецкий автономный округ \\
    91 & Республика Крым \\
    92 & г. Севастополь \\
    99 & Байконур \\
\end{xtabular}
\onecolumn

\subsection{ОКАТО}

Общероссийский классификатор объектов административно-территориального деления --- классификатор объектов административно-территориального деления Российской Федерации, предназначен для обеспечения достоверности, сопоставимости и автоматизированной обработки информации.

\begin{center}
\begin{longtable}{|r|p{.5\textwidth}|p{.2\textwidth}|}
    \hline
    \textbf{Код} & \textbf{Наименование} & \textbf{Ад\-ми\-нис\-тра\-тив\-ный центр} \\ \hline
    01 000 000 000 & Алтайский край & г Барнаул \\ \hline
    03 000 000 000 & Краснодарский край & г Краснодар \\ \hline
    04 000 000 000 & Красноярский край & г Красноярск \\ \hline
    05 000 000 000 & Приморский край & г Владивосток \\ \hline
    07 000 000 000 & Ставропольский край & г Ставрополь \\ \hline
    08 000 000 000 & Хабаровский край & г Хабаровск \\ \hline
    10 000 000 000 & Амурская область & г Благовещенск \\ \hline
    11 000 000 000 & Архангельская область & г Архангельск \\ \hline
    12 000 000 000 & Астраханская область & г Астрахань \\ \hline
    14 000 000 000 & Белгородская область & г Белгород \\ \hline
    15 000 000 000 & Брянская область & г Брянск \\ \hline
    17 000 000 000 & Владимирская область & г Владимир \\ \hline
    18 000 000 000 & Волгоградская область & г Волгоград \\ \hline
    19 000 000 000 & Вологодская область & г Вологда \\ \hline
    20 000 000 000 & Воронежская область & г Воронеж \\ \hline
    22 000 000 000 & Нижегородская область & г Нижний Новгород \\ \hline
    24 000 000 000 & Ивановская область & г Иваново \\ \hline
    25 000 000 000 & Иркутская область & г Иркутск \\ \hline
    26 000 000 000 & Республика Ингушетия & г Магас \\ \hline
    27 000 000 000 & Калининградская область & г Калининград \\ \hline
    28 000 000 000 & Тверская область & г Тверь \\ \hline
    29 000 000 000 & Калужская область & г Калуга \\ \hline
    30 000 000 000 & Камчатский край & г Петропавловск-Камчатский \\ \hline
    32 000 000 000 & Кемеровская область - Кузбасс & г Кемерово \\ \hline
    33 000 000 000 & Кировская область & г Киров \\ \hline
    34 000 000 000 & Костромская область & г Кострома \\ \hline
    35 000 000 000 & Республика Крым & г Симферополь \\ \hline
    36 000 000 000 & Самарская область & г Самара \\ \hline
    37 000 000 000 & Курганская область & г Курган \\ \hline
    38 000 000 000 & Курская область & г Курск \\ \hline
    40 000 000 000 & Город Санкт-Петербург город федерального значения &  \\ \hline	
    41 000 000 000 & Ленинградская область & г Санкт-Петербург \\ \hline
    42 000 000 000 & Липецкая область & г Липецк \\ \hline
    44 000 000 000 & Магаданская область & г Магадан \\ \hline
    45 000 000 000 & Город Москва столица Российской Федерации город федерального значения & \\ \hline
    46 000 000 000 & Московская область & г Москва \\ \hline
    47 000 000 000 & Мурманская область & г Мурманск \\ \hline
    49 000 000 000 & Новгородская область & г Великий Новгород \\ \hline
    50 000 000 000 & Новосибирская область & г Новосибирск \\ \hline
    52 000 000 000 & Омская область & г Омск \\ \hline
    53 000 000 000 & Оренбургская область & г Оренбург \\ \hline
    54 000 000 000 & Орловская область & г Орёл \\ \hline
    56 000 000 000 & Пензенская область & г Пенза \\ \hline
    57 000 000 000 & Пермский край & г Пермь \\ \hline
    58 000 000 000 & Псковская область & г Псков \\ \hline
    60 000 000 000 & Ростовская область & г Ростов-на-Дону \\ \hline
    61 000 000 000 & Рязанская область & г Рязань \\ \hline
    63 000 000 000 & Саратовская область & г Саратов \\ \hline
    64 000 000 000 & Сахалинская область & г Южно-Сахалинск \\ \hline
    65 000 000 000 & Свердловская область & г Екатеринбург \\ \hline
    66 000 000 000 & Смоленская область & г Смоленск \\ \hline
    67 000 000 000 & Город федерального значения Севастополь & \\ \hline
    68 000 000 000 & Тамбовская область & г Тамбов \\ \hline
    69 000 000 000 & Томская область & г Томск \\ \hline
    70 000 000 000 & Тульская область & г Тула \\ \hline
    71 000 000 000 & Тюменская область & г Тюмень \\ \hline
    73 000 000 000 & Ульяновская область & г Ульяновск \\ \hline
    75 000 000 000 & Челябинская область & г Челябинск \\ \hline
    76 000 000 000 & Забайкальский край & г Чита \\ \hline
    77 000 000 000 & Чукотский автономный округ & г Анадырь \\ \hline
    78 000 000 000 & Ярославская область & г Ярославль \\ \hline
    79 000 000 000 & Республика Адыгея (Адыгея) & г Майкоп \\ \hline
    80 000 000 000 & Республика Башкортостан & г Уфа \\ \hline
    81 000 000 000 & Республика Бурятия & г Улан-Удэ \\ \hline
    82 000 000 000 & Республика Дагестан & г Махачкала \\ \hline
    83 000 000 000 & Кабардино-Балкарская Республика & г Нальчик \\ \hline
    84 000 000 000 & Республика Алтай & г Горно-Алтайск \\ \hline
    85 000 000 000 & Республика Калмыкия & г Элиста \\ \hline
    86 000 000 000 & Республика Карелия & г Петрозаводск \\ \hline
    87 000 000 000 & Республика Коми & г Сыктывкар \\ \hline
    88 000 000 000 & Республика Марий Эл & г Йошкар-Ола \\ \hline
    89 000 000 000 & Республика Мордовия & г Саранск \\ \hline
    90 000 000 000 & Республика Северная Осетия-Алания & г Владикавказ \\ \hline
    91 000 000 000 & Карачаево-Черкесская Республика & г Черкесск \\ \hline
    92 000 000 000 & Республика Татарстан (Татарстан) & г Казань \\ \hline
    93 000 000 000 & Республика Тыва & г Кызыл \\ \hline
    94 000 000 000 & Удмуртская Республика & г Ижевск \\ \hline
    95 000 000 000 & Республика Хакасия & г Абакан \\ \hline
    96 000 000 000 & Чеченская Республика & г Грозный \\ \hline
    97 000 000 000 & Чувашская Республика - Чувашия & г Чебоксары \\ \hline
    98 000 000 000 & Республика Саха (Якутия) & г Якутск \\ \hline
    99 000 000 000 & Еврейская автономная область & г Биробиджан \\ \hline

\end{longtable}
\end{center}

\twocolumn[
\section{Описание ОКИН в соответствии с заданием}

\subsection{ОКИН: 03 --- Национальности}]

\begin{xtabular}{rp{.38\textwidth}}
    \hline
    \textbf{Код} & \textbf{Наименование} \\ \hline
    001 & Русские \\
    003 & Поморы \\
    004 & Абазины \\
    005 & Абхазы \\
    006 & Аварцы \\
    007 & Андийцы \\
    008 & Арчинцы \\
    009 & Ахвахцы \\
    010 & Багулалы \\
    011 & Бежтинцы \\
    012 & Ботлихцы \\
    013 & Гинухцы \\
    014 & Годоберинцы \\
    015 & Гунзибцы \\
    016 & Дидойцы \\
    017 & Каратинцы \\
    018 & Тиндалы \\
    019 & Хваршины \\
    020 & Чамалалы \\
    021 & Агулы \\
    022 & Адыгейцы \\
    023 & Азербайджанцы \\
    024 & Алеуты \\
    026 & Теленгиты \\
    027 & Тубалары \\
    028 & Челканцы \\
    029 & Американцы \\
    030 & Арабы \\
    031 & Арабы среднеазиатские \\
    032 & Армяне \\
    033 & Черкесогаи \\
    034 & Ассирийцы \\
    035 & Афганцы \\
    036 & Балкарцы \\
    037 & Башкиры \\
    038 & Белорусы \\
    039 & Бесермяне \\
    040 & Болгары \\
    041 & Боснийцы \\
    042 & Британцы \\
    043 & Буряты \\
    044 & Венгры \\
    045 & Вепсы \\
    046 & Водь \\
    047 & Вьетнамцы \\
    048 & Гагаузы \\
    049 & Горские евреи \\
    050 & Греки \\
    051 & Греки-урумы \\
    052 & Грузинские евреи \\
    053 & Грузины \\
    054 & Аджарцы \\
    055 & Ингилойцы \\
    056 & Лазы \\
    057 & Мегрелы \\
    058 & Сваны \\
    059 & Даргинцы \\
    060 & Кайтагцы \\
    061 & Кубачинцы \\
    062 & Долганы \\
    063 & Дунгане \\
    064 & Евреи \\
    065 & Езиды \\
    066 & Ижорцы \\
    067 & Ингуши \\
    068 & Индийцы \\
    069 & Испанцы \\
    070 & Итальянцы \\
    071 & Ительмены \\
    072 & Кабардинцы \\
    073 & Казахи \\
    074 & Калмыки \\
    075 & Камчадалы \\
    076 & Караимы \\
    077 & Каракалпаки \\
    078 & Карачаевцы \\
    079 & Карелы \\
    080 & Кереки \\
    081 & Кеты \\
    082 & Юги \\
    083 & Киргизы \\
    084 & Китайцы \\
    085 & Коми \\
    086 & Коми-ижемцы \\
    087 & Коми-пермяки \\
    088 & Корейцы \\
    089 & Коряки \\
    090 & Алюторцы \\
    091 & Крымские татары \\
    092 & Крымчаки \\
    093 & Кубинцы \\
    094 & Кумандинцы \\
    095 & Кумыки \\
    096 & Курды \\
    097 & Курманч \\
    098 & Лакцы \\
    099 & Латыши \\
    100 & Латгальцы \\
    101 & Лезгины \\
    102 & Литовцы \\
    103 & Македонцы \\
    104 & Манси \\
    105 & Марийцы \\
    106 & Горные марийцы \\
    107 & Лугово-восточные марийцы \\
    108 & Молдаване \\
    109 & Монголы \\
    110 & Мордва \\
    111 & Мордва-мокша \\
    112 & Мордва-эрзя \\
    113 & Нагайбаки \\
    115 & Нганасаны \\
    116 & Негидальцы \\
    117 & Немцы \\
    118 & Меннониты \\
    119 & Ненцы \\
    120 & Нивхи \\
    121 & Ногайцы \\
    122 & Карагаши \\
    123 & Орочи \\
    124 & Осетины \\
    125 & Осетины-дигорцы \\
    126 & Осетины-иронцы \\
    127 & Пакистанцы \\
    128 & Памирцы \\
    129 & Персы \\
    130 & Поляки \\
    131 & Румыны \\
    132 & Русины \\
    133 & Рутульцы \\
    134 & Саамы \\
    135 & Селькупы \\
    136 & Сербы \\
    137 & Словаки \\
    138 & Словенцы \\
    139 & Сойоты \\
    140 & Среднеазиатские евреи \\
    141 & Табасараны \\
    142 & Таджики \\
    143 & Тазы \\
    144 & Талыши \\
    145 & Татары \\
    146 & Астраханские татары \\
    147 & Кряшены \\
    148 & Мишари \\
    149 & Сибирские татары \\
    150 & Таты \\
    151 & Телеуты \\
    152 & Тофалары (тофа) \\
    153 & Тувинцы \\
    154 & Тувинцы-тоджинцы \\
    155 & Турки \\
    156 & Турки-месхетинцы \\
    157 & Туркмены \\
    158 & Удины \\
    159 & Удмурты \\
    160 & Удэгейцы \\
    161 & Узбеки \\
    162 & Уйгуры \\
    163 & Уйльта (ороки) \\
    164 & Украинцы \\
    165 & Ульчи \\
    166 & Финны \\
    167 & Финны-ингерманландцы \\
    168 & Французы \\
    169 & Хакасы \\
    170 & Ханты \\
    171 & Хемшилы \\
    172 & Хорваты \\
    173 & Цахуры \\
    174 & Цыгане \\
    175 & Цыгане среднеазиатские \\
    176 & Черкесы \\
    177 & Черногорцы \\
    178 & Чехи \\
    179 & Чеченцы \\
    180 & Чеченцы-аккинцы \\
    181 & Чуванцы \\
    182 & Чуваши \\
    183 & Чукчи \\
    184 & Чулымцы \\
    185 & Шапсуги \\
    186 & Шорцы \\
    187 & Эвенки \\
    188 & Эвены (ламуты) \\
    189 & Энцы \\
    190 & Эскимосы \\
    191 & Эстонцы \\
    192 & Сету (сето) \\
    193 & Юкагиры \\
    194 & Якуты (саха) \\
    195 & Японцы \\
    196 & Австралийцы \\
    197 & Австрийцы \\
    198 & Айны \\
    199 & Албанцы \\
    200 & Ангольцы \\
    201 & Аргентинцы \\
    202 & Бангладешцы \\
    203 & Баски \\
    204 & Бельгийцы \\
    205 & Бенинцы \\
    206 & Берберы \\
    207 & Бирманцы \\
    208 & Бисау-гвинейцы \\
    209 & Боливийцы \\
    210 & Ботсванцы \\
    211 & Бразильцы \\
    212 & Булгары \\
    213 & Буркинабцы \\
    214 & Бурундийцы \\
    215 & Венесуэльцы \\
    216 & Габонцы \\
    217 & Гаитяне \\
    218 & Гамбийцы \\
    219 & Ганцы \\
    220 & Гватемальцы \\
    221 & Гвинейцы \\
    222 & Голландцы \\
    223 & Гондурасцы \\
    224 & Дагестанцы \\
    225 & Датчане \\
    226 & Дауры \\
    227 & Джекцы \\
    228 & Джибутинцы \\
    229 & Доминиканцы \\
    230 & Замбийцы \\
    231 & Зимбабвийцы \\
    232 & Израильтяне \\
    233 & Индонезийцы \\
    234 & Ирландцы \\
    235 & Исландцы \\
    236 & Кабоверденцы \\
    237 & Камасинцы \\
    238 & Камбоджийцы \\
    239 & Камерунцы \\
    240 & Канадцы \\
    241 & Кенийцы \\
    242 & Киприоты \\
    243 & Кистины \\
    244 & Колумбийцы \\
    245 & Коморцы \\
    246 & Конголезцы \\
    247 & Костариканцы \\
    248 & Котдивуарцы \\
    249 & Ланкийцы \\
    250 & Лаосцы \\
    251 & Лесотцы \\
    252 & Либерийцы \\
    253 & Лихтенштейнцы \\
    254 & Люксембуржцы \\
    255 & Маврикийцы \\
    256 & Малавийцы \\
    257 & Малагасийцы \\
    258 & Малайцы \\
    259 & Малийцы \\
    260 & Мальдивцы \\
    261 & Маньчжуры \\
    262 & Мексиканцы \\
    263 & Мозамбикцы \\
    264 & Монегаски \\
    265 & Намибийцы \\
    266 & Непальцы \\
    267 & Нигерийцы \\
    268 & Нигерцы \\
    269 & Никарагуанцы \\
    270 & Новогвинейцы \\
    271 & Новозеландцы \\
    272 & Норвежцы \\
    273 & Панамцы \\
    274 & Парагвайцы \\
    275 & Перуанцы \\
    276 & Полинезийцы \\
    277 & Португальцы \\
    278 & Пуэрториканцы \\
    279 & Россияне \\
    280 & Руандийцы \\
    281 & Сальвадорцы \\
    282 & Сантомийцы \\
    283 & Сенегальцы \\
    284 & Сибо \\
    285 & Сомалийцы \\
    286 & Суданцы \\
    287 & Сьерралеонцы \\
    288 & Таиландцы \\
    289 & Танзанийцы \\
    290 & Тоголезцы \\
    291 & Тонганцы \\
    292 & Тюрки \\
    293 & Угандийцы \\
    294 & Уругвайцы \\
    295 & Филиппинцы \\
    296 & Чадцы \\
    297 & Чилийцы \\
    298 & Шведы \\
    299 & Швейцарцы \\
    300 & Эквадорцы \\
    301 & Экваторианцы \\
    302 & Эритрейцы \\
    303 & Эфиопы \\
    304 & Югославы \\
    305 & Южноафриканцы \\
    306 & Ямайцы \\
    307 & Прочие национальности \\
\end{xtabular}
\onecolumn

\subsection{ОКИН: 15 --- Участие в войне}

\begin{center}
    \begin{tabular}{rp{.8\textwidth}}
        \hline
        \textbf{Код} & \textbf{Наименование} \\ \hline
        1 & Участвовал в Великой Отечественной войне \\
        2 & Участвовал в боевых действиях на территориях других государств \\
        3 & Не участвовал в войне \\
    \end{tabular}
\end{center}

\subsection{ОКИН: 16 --- Отношение к военной службе}

\begin{center}
    \begin{tabular}{rp{.8\textwidth}}
        \hline
        \textbf{Код} & \textbf{Наименование} \\ \hline
        1 & Военнослужащий \\
        2 & Военнообязанный \\
        3 & Невоеннообязанный \\
        4 & Призывник \\
        5 & Служащий таможенных органов \\
    \end{tabular}
\end{center}

\subsection{ОКИН: 17 --- Воинские звания}

\begin{center}
    \begin{longtable}{rp{.8\textwidth}}
        \hline
        \textbf{Код} & \textbf{Наименование} \\ \hline
        01 & Рядовой (матрос) \\
        02 & Ефрейтор (старший матрос) \\
        03 & Младший сержант (старшина 2-й статьи) \\
        04 & Сержант (старшина 1-й статья) \\
        05 & Старший сержант (главный старшина) \\
        06 & Старшина (главный корабельный старшина) \\
        07 & Прапорщик (мичман) \\
        08 & Старший прапорщик (старший мичман) \\
        09 & Младший лейтенант и ему равные \\
        10 & Лейтенант и ему равные \\
        11 & Старший лейтенант и ему равные \\
        12 & Капитан (капитан-лейтенант) и им равные \\
        13 & Майор (капитан 3-го ранга) и им равные \\
        14 & Подполковник (капитан 2-го ранга) и им равные \\
        15 & Полковник (капитан 1-го ранга) и им равные \\
        16 & Генерал-майор (контр-адмирал) и им соответствующие \\
        17 & Генерал-лейтенант (вице-адмирал) и им соответствующие \\
        18 & Генерал-полковник (адмирал) и им соответствующие \\
        19 & Генерал армии (адмирал флота, маршал рода войск) и им соответствующие \\
        21 & Главный маршал рода войск \\
        23 & Маршал Советского Союза (Адмирал Флота Советского Союза) \\
        31 & Действительный государственный советник таможенной службы \\
        32 & Государственный советник таможенной службы I ранга \\
        33 & Государственный советник таможенной службы II ранга \\
        34 & Государственный советник таможенной службы III ранга \\
        35 & Советник таможенной службы I ранга \\
        36 & Советник таможенной службы II ранга \\
        37 & Советник таможенной службы III ранга \\
        38 & Инспектор таможенной службы I ранга \\
        39 & Инспектор таможенной службы II ранга \\
        40 & Инспектор таможенной службы III ранга \\
    \end{longtable}
\end{center}

\subsection{ОКИН: 20 --- Виды занятости}

\begin{center}
    \begin{tabular}{rp{.8\textwidth}}
        \hline
        \textbf{Код} & \textbf{Наименование} \\ \hline
        1 & Работающий на постоянной работе \\
        2 & Работающий на временной работе \\
        3 & Работающий на сезонной работе \\
        4 & Работающий по срочному трудовому договору \\
        5 & Неработающий \\
        6 & Безработный \\
        7 & Безработный, зарегистрированный в органах службы занятости \\
    \end{tabular}
\end{center}

\subsection{ОКИН: 21 --- Стаж работы}

\begin{center}
    \begin{longtable}{rp{.8\textwidth}}
        \hline
        \textbf{Код} & \textbf{Наименование} \\ \hline
        01 & Стаж работы/менее месяца \\
        02 & - от 1 до 3 мес. \\
        03 & - от 1 до 6 мес. \\
        04 & - от 7 до 11 мес. \\
        05 & - до 1 года \\
        06 & - от 1 года до 2 лет \\
        07 & - не менее 2 лет \\
        08 & - 2 года и более лет \\
        09 & - от 2 лет до 3 лет \\
        10 & - от 1 года до 5 лет \\
        11 & - до 3 лет \\
        12 & - от 3 лет до 5 лет \\
        13 & - от 1 года \\
        14 & - менее 5 лет \\
        15 & - 5 лет \\
        16 & - от 5 лет и выше \\
        17 & - от 5 до 10 лет \\
        18 & - от 5 до 15 лет \\
        19 & - 6 лет \\
        20 & - 7 лет \\
        21 & - 8 лет \\
        22 & - 9 лет \\
        23 & - 10 лет \\
        24 & - 10 лет и выше \\
        25 & - свыше 10 лет \\
        26 & - от 10 до 15 лет \\
        27 & - от 10 до 25 лет \\
        28 & - 11 лет \\
        29 & - 12 лет \\
        30 & - 13 лет \\
        31 & - 14 лет \\
        32 & - 15 лет \\
        33 & - 15 лет и выше \\
        34 & - 16 лет \\
        35 & - 17 лет \\
        36 & - 18 лет \\
        37 & - 19 лет \\
        38 & - 20 лет \\
        39 & - 21 год \\
        40 & - 22 года \\
        41 & - 23 года \\
        42 & - 24 года \\
        43 & - 25 лет \\
        44 & - 25 и более лет \\
        45 & - 30 и более лет \\
        46 & - 35 и более лет \\
        47 & - 40 и более лет \\
        48 & - 45 и более лет \\
        49 & Стаж работы не указан \\
    \end{longtable}
\end{center}

\subsection{ОКИН: 24 --- Виды отпусков}

\begin{center}
    \begin{longtable}{rp{.8\textwidth}}
        \hline
        \textbf{Код} & \textbf{Наименование} \\ \hline
        01 & Ежегодный отпуск \\
        02 & Отпуск без сохранения заработной платы \\
        03 & Отпуск по беременности и родам \\
        04 & Отпуск женщинам, усыновившим новорожденных детей непосредственно из родильного дома \\
        05 & Дополнительный отпуск рабочим и служащим, занятым на работах с вредными и (или) опасными условиями труда \\
        06 & Дополнительный отпуск рабочим и служащим, занятым в отдельных отраслях народного хозяйства и имеющим продолжительный стаж работы на одном предприятии, в организации \\
        07 & Дополнительным отпуск работникам с ненормированным рабочим днем \\
        08 & Дополнительный отпуск рабочим и служащим, работающим в районах Крайнего Севера и приравненных к ним местностях \\
        09 & Дополнительный отпуск работающим на территориях в районах загрязнения от аварий на ЧАЭС \\
        10 & Дополнительный отпуск участникам ликвидации аварий на ЧАЭС и других радиационных аварий \\
        12 & Отпуск для сдачи экзаменов в вечерних (сменных) общеобразовательных школах \\
        13 & Отпуск в связи с обучением в вечерних профессионально-технических училищах профессионального образования \\
        14 & Дополнительный отпуск для сдачи вступительных экзаменов в аспирантуру \\
        15 & Дополнительным ежегодный отпуск аспирантам \\
        16 & Отпуск без сохранения заработной платы для сдачи вступительных экзаменов в высшие и средние учреждения профессионального образования \\
        17 & Отпуск в связи с обучением в вечерних и заочных высших и средних учреждениях профессионального образования \\
        18 & Отпуск для ознакомления с работой по избранной специальности и подготовки материалов к дипломному проекту \\
        20 & Дополнительный отпуск за донорство \\
        21 & Творческий отпуск \\
        22 & Частично оплачиваемый отпуск женщинам, имеющим детей в возрасте до 1,5 лет \\
        23 & Дополнительный отпуск без сохранения заработной платы женщинам, имеющим детей в возрасте до 3 лет \\
        24 & Дополнительный отпуск государственного служащего за стаж работы \\
    \end{longtable}
\end{center}

\subsection{ОКИН: 30 --- Образование}

\begin{center}
    \begin{tabular}{rp{.8\textwidth}}
        \hline
        \textbf{Код} & \textbf{Наименование} \\ \hline
        01 & Дошкольное образование \\
        02 & Начальное общее образование \\
        03 & Основное общее образование \\
        04 & Среднее общее образование \\
        05 & Среднее профессиональное образование \\
        06 & Высшее образование - бакалавриат \\
        07 & Высшее образование - специалитет, магистратура \\
        08 & Высшее образование - подготовка кадров высшей квалификации \\
        10 & Профессиональное обучение \\
        11 & Дополнительное образование детей и взрослых \\
        12 & Дополнительное профессиональное образование \\
    \end{tabular}
\end{center}

\subsection{ОКИН: 31 --- Образовательные учреждения}

\begin{center}
    \begin{longtable}{rp{.8\textwidth}}
        \hline
        \textbf{Код} & \textbf{Наименование} \\ \hline
        01 & Дошкольное образовательное учреждение \\
        10 & Образовательное учреждение \\
        12 & Начальная общеобразовательная школа \\
        13 & Основная школа \\
        14 & Семилетняя школа \\
        15 & Восьмилетняя школа \\
        16 & Девятилетняя школа \\
        17 & Гимназия \\
        18 & Лицей \\
        20 & Средняя (полная) общеобразовательная школа с углубленным изучением отдельных предметов \\
        21 & Вечерняя (сменная) общеобразовательная школа \\
        29 & Средняя (полная) общеобразовательная школа \\
        30 & Образовательное учреждение начального профессионального образования \\
        31 & Профессионально-техническое училище \\
        40 & Образовательное учреждение среднего профессионального образования \\
        41 & Училище \\
        42 & Техникум \\
        43 & Колледж \\
        50 & Образовательное учреждение высшего профессионального образования (высшее учебное заведение) \\
        51 & Высшее училище \\
        52 & Школа-студия \\
        53 & Консерватория \\
        54 & Институт \\
        55 & Университет \\
        56 & Академия \\
        61 & Негосударственное (частное) образовательное учреждение \\
        62 & Специальное (коррекционное) образовательное учреждение для обучающихся, воспитанников с отклонениями и развитии \\
        63 & Учреждение для детей-сирот и детей, оставшихся без попечения родителей \\
        69 & Другие учреждения, осуществляющие образовательный процесс \\
    \end{longtable}
\end{center}

\subsection{ОКИН: 32 --- Типы образовательных организаций}

\begin{center}
    \begin{longtable}{rp{.8\textwidth}}
        \hline
        \textbf{Код} & \textbf{Наименование} \\ \hline
        01 & Образовательные организации, в том числе: \\
        02 & дошкольная образовательная организация \\
        03 & общеобразовательная организация \\
        04 & профессиональная образовательная организация \\
        05 & образовательная организация высшего образования \\
        06 & организация дополнительного образования \\
        07 & организация дополнительного профессионального образования \\
        08 & Организации, осуществляющие обучение, в том числе: \\
        09 & научные организации \\
        10 & организации для детей-сирот и детей, оставшихся без попечения родителей \\
        11 & организации, осуществляющие лечение, оздоровление и (или) отдых \\
        12 & организации, осуществляющие социальное обслуживание \\
        13 & другие \\
        14 & Организация, расположенная на территории инновационного центра "Сколково" и осуществляющая образовательную деятельность \\
        15 & Индивидуальные предприниматели, осуществляющие образовательную деятельность \\
    \end{longtable}
\end{center}

\subsection{ОКИН: 33 --- Формы обучения}

\begin{center}
    \begin{tabular}{rp{.8\textwidth}}
        \hline
        \textbf{Код} & \textbf{Наименование} \\ \hline
        1 & Очное \\
        2 & Очно-заочное (вечернее) \\
        3 & Заочное \\
        4 & Экстернат \\
        5 & Семейное образование \\
        6 & Самообразование \\
    \end{tabular}
\end{center}

\subsection{ОКИН: 34 --- Подготовка, переподготовка и повышение квалификации кадров}

\begin{center}
    \begin{longtable}{rp{.8\textwidth}}
        \hline
        \textbf{Код} & \textbf{Наименование} \\ \hline
        01 & Докторантура \\
        02 & Аспирантура, ординатура, адъюнктура \\
        05 & Институт повышения квалификации министерства (ведомства) \\
        07 & Факультет повышения квалификации при высшем учебном заведении \\
        08 & Факультет повышения квалификации при среднем профессиональном учебном заведении \\
        09 & Институт усовершенствования \\
        10 & Курсы повышения квалификации при министерстве (ведомстве) \\
        11 & Курсы повышения квалификации при предприятиях, научно-исследовательских и проектно-конструкторских организациях, высших и средних учреждениях профессионального образования, институтах повышения квалификации и их филиалах \\
        21 & Училище \\
        32 & Курсы целевого назначения \\
        33 & Школа по изучению передовых методов труда \\
        34 & Школа мастеров \\
        41 & Народный университет \\
        42 & Негосударственное (частное) образовательное учреждение \\
        43 & Учебный центр службы занятости \\
        49 & Другие формы повышения квалификации рабочих и других работников \\
    \end{longtable}
\end{center}

\subsection{ОКИН: 35 --- Ученые степени}

\begin{center}
    \begin{tabular}{rp{.8\textwidth}}
        \hline
        \textbf{Код} & \textbf{Наименование} \\ \hline
        1 & Доктор наук \\
        2 & Кандидат наук \\
    \end{tabular}
\end{center}

\subsection{ОКИН: 36 --- Ученые звания}

\begin{center}
    \begin{tabular}{rp{.8\textwidth}}
        \hline
        \textbf{Код} & \textbf{Наименование} \\ \hline
        01 & Академик Российской Академии Наук \\
        02 & Академик международной академии наук \\
        03 & Академик отраслевой академии наук \\
        04 & Член-корреспондент Российской Академии Наук \\
        05 & Член-корреспондент международной академии наук \\
        06 & Член-корреспондент отраслевой академии наук \\
        07 & Профессор \\
        08 & Доцент \\
        09 & Старший научный сотрудник \\
        10 & Младший научный сотрудник \\
        11 & Ассистент \\
        12 & Член зарубежной академии наук \\
    \end{tabular}
\end{center}

\subsection{ОКИН: 38 --- Виды домохозяйств}

\begin{center}
    \begin{tabular}{rp{.8\textwidth}}
        \hline
        \textbf{Код} & \textbf{Наименование} \\ \hline
        1 & Частное домохозяйство \\
        2 & Коллективное домохозяйство (институциональное население) \\
        3 & Домохозяйство бездомных (лиц без определенного места жительства) \\
    \end{tabular}
\end{center}

\subsection{ОКИН: 40 --- Типы частных домохозяйств}

\begin{center}
    \begin{longtable}{rp{.8\textwidth}}
        \hline
        \textbf{Код} & \textbf{Наименование} \\ \hline
        10 & Домохозяйство, состоящее из одного человека \\
        20 & Домохозяйство, состоящее из одной супружеской пары \\
        21 & Домохозяйство, состоящее из супружеской пары с детьми и без детей \\
        22 & Домохозяйство, состоящее из супружеской пары с детьми и без детей и одного из родителей супругов \\
        23 & Домохозяйство, состоящее из супружеской пары с детьми и без детей и матери с детьми \\
        24 & Домохозяйство, состоящее из супружеской пары с детьми и без детей и отца с детьми \\
        25 & Домохозяйство, состоящее из супружеской пары с детьми и без детей, с одним из родителей супругов (или без него), с матерью (отцом) с детьми (или без нее (него)) и с прочими родственниками или неродственниками \\
        30 & Домохозяйство, состоящее из двух супружеских пар \\
        31 & Домохозяйство, состоящее из супружеской пары с детьми и без детей и обоими родителями одного из супругов с детьми и без детей \\
        32 & Домохозяйство, состоящее из супружеской пары с детьми и без детей, с обоими родителями одного из супругов с детьми и без детей и прочих родственников \\
        33 & Домохозяйство, состоящее из супружеской пары с детьми и без детей, с обоими родителями одного из супругов с детьми и без детей, прочих родственников (или без них) и неродственников \\
        34 & Домохозяйство, состоящее из двух супружеских пар с детьми и без детей, с родственниками, неродственниками (или без них) \\
        40 & Домохозяйство, состоящее из трех и более супружеских пар с детьми и без детей, с родственниками, неродственниками (или без них) \\
        50 & Домохозяйство, состоящее из матери с детьми \\
        60 & Домохозяйство, состоящее из отца с детьми \\
        70 & Домохозяйство, состоящее из матери с детьми и одним из родителей матери (отца) \\
        80 & Домохозяйство, состоящее из отца с детьми и одним из родителей отца (матери) \\
        90 & Домохозяйство, состоящее из матери с детьми, с одним из родителей матери (отца) (или без него), с прочими родственниками или неродственниками \\
        100 & Домохозяйство, состоящее из отца с детьми, с одним из родителей отца (матери) (или без него), с прочими родственниками или неродственниками \\
        110 & Домохозяйство, состоящее из лиц, не связанных родством \\
        120 & Прочие домохозяйства \\
    \end{longtable}
\end{center}

\subsection{ОКИН: 41 --- Статьи дохода домохозяйства}

\begin{center}
    \begin{longtable}{rp{.8\textwidth}}
        \hline
        \textbf{Код} & \textbf{Наименование} \\ \hline
        01 & Заработная плата деньгами и натурой/обследуемого члена домохозяйства \\
        02 & - других членов домохозяйства \\
        03 & Премия, единовременное поощрение и вознаграждение из фонда материального поощрения/обследуемого члена домохозяйства \\
        04 & - других членов домохозяйства \\
        05 & Премия, полученная из фонда заработной платы \\
        11 & Пособие/по социальному страхованию \\
        12 & - многодетным и одиноким матерям \\
        13 & - на детей малоимущим домохозяйствам \\
        14 & - по безработице \\
        15 & - прочее \\
        21 & Дотация/на путевку в санаторий, дом отдыха, оздоровительный лагерь \\
        22 & - на содержание детей в детских дошкольных учреждениях \\
        23 & - на общественное питание \\
        26 & Компенсация проживающим на радиационно загрязненных территориях \\
        27 & Денежная выплата на питание участникам ликвидации аварии на ЧАЭС и других радиационных аварий \\
        28 & Доплата работающим на территориях, подверженных радиационным загрязнениям \\
        31 & Стипендия \\
        32 & Пенсия \\
        33 & Суточные, полученные при командировках, подъемные \\
        37 & Поступления/по госзаймам \\
        38 & - от госстрахования \\
        39 & - за кустарно-ремесленные работы \\
        40 & Другие поступления из общественных фондов потребления \\
        51 & Доход от личного хозяйства \\
        52 & Доход от личного подсобного хозяйства \\
        57 & Доход от предпринимательской деятельности \\
        58 & Доход от индивидуально-трудовой деятельности \\
        61 & Доход от акций, облигаций и других ценных бумаг \\
        62 & Доход от вкладов в банки и другие кредитно-финансовые организации \\
        65 & Доход от изменения котировки акций и валют \\
        68 & Доход при наследовании и дарении \\
        71 & Доход от сдачи жилья \\
        75 & Доход, полученный за работу (услугу) \\
        81 & Ссуда и взятие денег в долг \\
        85 & Остаток денег у домохозяйства \\
    \end{longtable}
\end{center}

\subsection{ОКИН: 43 --- Источники средств к существованию}

\begin{center}
    \begin{longtable}{rp{.8\textwidth}}
        \hline
        \textbf{Код} & \textbf{Наименование} \\ \hline
        01 & Трудовая деятельность, включая работу по совместительству \\
        02 & Личное подсобное хозяйство \\
        03 & Стипендия \\
        04 & Пенсия (кроме пенсии по инвалидности) \\
        05 & Пенсия по инвалидности \\
        06 & Пособие (кроме пособия по безработице) \\
        07 & Пособие по безработице \\
        08 & Другой вид государственного обеспечения \\
        09 & Сбережения; дивиденды; проценты \\
        10 & Сдача внаем или в аренду имущества; доход от патентов, авторских прав \\
        11 & Иждивение: помощь других лиц; алименты \\
        12 & Иной источник \\
    \end{longtable}
\end{center}

\subsection{ОКИН: 45 --- Группы граждан, нуждающихся в жилой площади}

\begin{center}
    \begin{longtable}{rp{.8\textwidth}}
        \hline
        \textbf{Код} & \textbf{Наименование} \\ \hline
        01 & Лица, которые имеют на каждого члена семьи доход ниже уровня, установленного Правительством России \\
        02 & Участники Великой Отечественной войны и приравненные к ним лица \\
        03 & Лица, страдающие тяжелыми формами некоторых хронических заболеваний, перечисленных в списке заболеваний, утвержденном Правительством России \\
        04 & Инвалиды и семьи, имеющие детей-инвалидов, признанные нуждающимися в улучшении жилищных условий \\
        05 & Инвалиды Великой Отечественной войны \\
        06 & Лица, награжденные знаком <<Жителю блокадного Ленинграда>> \\
        07 & Члены семей инвалидов и участников Великой Отечественной войны \\
        08 & Несовершеннолетние узники фашизма \\
        09 & Несовершеннолетние узники фашизма, ставшие инвалидами \\
        10 & Лица, работавшие на объектах противовоздушной обороны, местной противовоздушной обороны, на строительстве оборонительных сооружений, военно-морских баз, аэродромов и других военных объектов в пределах тыловых границ действующих фронтов, операционных зон действующих флотов, на прифронтовых участках железных и автомобильных дорог, члены экипажей судов транспортного флота, интернированные в начале Великой Отечественной войны в портах других государств \\
        11 & Инвалиды боевых действий \\
        12 & Ветераны боевых действий \\
        13 & Члены семей инвалидов и ветеранов боевых действий \\
    \end{longtable}
\end{center}

\subsection{ОКИН: 50 --- Причины миграции населения}

\begin{center}
    \begin{tabular}{rp{.8\textwidth}}
        \hline
        \textbf{Код} & \textbf{Наименование} \\ \hline
        01 & В связи с учебой \\
        02 & В связи с работой \\
        03 & Возвращение к прежнему месту жительства \\
        04 & Обострение межнациональных отношений \\
        05 & Обострение криминогенной обстановки \\
        06 & Экологическое неблагополучие \\
        07 & Неблагоприятные для здоровья природно-климатические условия \\
        08 & Причины личного, семейного характера \\
        10 & Стихийные бедствия \\
        11 & Вооруженный конфликт \\
        12 & К месту прохождения воинской службы \\
        13 & В командировку \\
        14 & На отдых \\
        15 & В туристическую поездку \\
        19 & Прочие причины \\
        081 & И связи с переменой места работы супруга (супруги) \\
        082 & В связи с вступлением в брак \\
        083 & К детям \\
        084 & К родителям \\
        191 & Приобретение жилья (покупка, наследование и т.п.) \\
    \end{tabular}
\end{center}

\subsection{ОКИН: 55 --- Государственные награды Российской Федерации}

\begin{center}
    \begin{longtable}{rp{.8\textwidth}}
        \hline
        \textbf{Код} & \textbf{Наименование} \\ \hline
        2010 & Высшие звания Российской Федерации \\
        2020 & Звание Героя Российской Федерации \\
        2022 & Звание Героя Труда Российской Федерации \\
        2024 & Ордена Российской Федерации \\
        2025 & Орден Святого апостола Андрея Первозванного \\
        2027 & Орден Святого апостола Андрея Первозванного с мечами \\
        2030 & Орден "За заслуги перед Отечеством" I степени \\
        2031 & Орден "За заслуги перед Отечеством" I степени с мечами \\
        2032 & Орден "За заслуги перед Отечеством" II степени \\
        2033 & Орден "За заслуги перед Отечеством" III степени \\
        2034 & Орден "За заслуги перед Отечеством" IV степени \\
        2035 & Орден "За заслуги перед Отечеством" II степени с мечами \\
        2036 & Орден "За заслуги перед Отечеством" III степени с мечами \\
        2037 & Орден "За заслуги перед Отечеством" IV степени с мечами \\
        2040 & Орден Святого Георгия I степени \\
        2041 & Орден Святого Георгия II степени \\
        2042 & Орден Святого Георгия III степени \\
        2043 & Орден Святого Георгия IV степени \\
        2050 & Орден Дружбы \\
        2070 & Орден Мужества \\
        2075 & Орден "За военные заслуги" \\
        2085 & Орден "За морские заслуги" \\
        2087 & Орден Пирогова \\
        2090 & Орден Почета \\
        2100 & Орден Суворова \\
        2115 & Орден Ушакова \\
        2130 & Орден Кутузова \\
        2140 & Орден Святой великомученицы Екатерины \\
        2150 & Орден Александра Невского \\
        2160 & Орден Нахимова \\
        2400 & Орден Жукова \\
        2500 & Орден "Родительская слава" \\
        3000 & Медали Российской Федерации \\ \\
        3080 & Медаль "За отвагу" \\
        3089 & Медаль "За отвагу при пожаре" \\
        3090 & Медаль "За спасение погибавших" \\
        3091 & Медаль Луки Крымского \\
        3140 & Медаль "За отличие в охране государственной границы" \\
        3170 & Медаль "За отличие в охране общественного порядка" \\
        3210 & Медаль Нестерова \\
        3245 & Медаль Пушкина \\
        3260 & Медаль Ушакова \\
        3270 & Медаль Суворова \\
        3280 & Медаль "Защитнику свободной России" \\
        3301 & Медаль ордена "За заслуги перед Отечеством" I степени \\
        3302 & Медаль ордена "За заслуги перед Отечеством" II степени \\
        3310 & Медаль ордена "За заслуги перед Отечеством" I степени с мечами \\
        3320 & Медаль ордена "За заслуги перед Отечеством" II степени с мечами \\
        3330 & Медаль "За развитие железных дорог" \\
        3332 & Медаль "За заслуги в освоении атомной энергии" \\
        3340 & Медаль "За заслуги в освоении космоса" \\
        3350 & Медаль ордена "Родительская слава" \\
        3482 & Медаль "За труды по сельскому хозяйству" \\
        3510 & Медаль Жукова \\
        4000 & Знаки отличия Российской Федерации \\
        4030 & Знак отличия - Георгиевский Крест I степени \\
        4031 & Знак отличия - Георгиевский Крест II степени \\
        4032 & Знак отличия - Георгиевский Крест III степени \\
        4033 & Знак отличия - Георгиевский Крест IV степени \\
        4040 & Знак отличия "За благодеяние" \\
        4043 & Знак отличия «За наставничество» \\
        4050 & Знак отличия "За безупречную службу" L лет \\
        4090 & Знак отличия "За безупречную службу" XL лет \\
        4130 & Знак отличия "За безупречную службу" XXX лет \\
        4150 & Знак отличия "За безупречную службу" XXV лет \\
        4170 & Знак отличия "За безупречную службу" XX лет \\
        4180 & Знак отличия "За безупречную службу" XX лет на георгиевской ленте \\
        4200 & Знак отличия "За безупречную службу" XV лет на георгиевской ленте \\
        6000 & Почетные звания Российской Федерации \\
        6010 & Народный артист Российской Федерации \\
        6040 & Народный архитектор Российской Федерации \\
        6060 & Народный учитель Российской Федерации \\
        6080 & Народный художник Российской Федерации \\
        6110 & Заслуженный артист Российской Федерации \\
        6120 & Заслуженный архитектор Российской Федерации \\
        6135 & Заслуженный военный специалист Российской Федерации \\
        6140 & Заслуженный врач Российской Федерации \\
        6143 & Заслуженный военный летчик Российской Федерации \\
        6146 & Заслуженный военный штурман Российской Федерации \\
        6150 & Заслуженный геолог Российской Федерации \\
        6160 & Заслуженный деятель искусств Российской Федерации \\
        6170 & Заслуженный деятель науки Российской Федерации \\
        6173 & Заслуженный журналист Российской Федерации \\
        6190 & Заслуженный землеустроитель Российской Федерации \\
        6192 & Заслуженный изобретатель Российской Федерации \\
        6230 & Заслуженный конструктор Российской Федерации \\
        6240 & Заслуженный лесовод Российской Федерации \\
        6243 & Заслуженный летчик-испытатель Российской Федерации \\
        6246 & Заслуженный мастер производственного обучения Российской Федерации \\
        6260 & Заслуженный машиностроитель Российской Федерации \\
        6280 & Заслуженный металлург Российской Федерации \\
        6283 & Заслуженный метеоролог Российской Федерации \\
        6305 & Заслуженный пилот Российской Федерации \\
        6308 & Заслуженный работник атомной промышленности Российской Федерации \\
        6315 & Заслуженный работник высшей школы Российской Федерации \\
        6320 & Заслуженный работник геодезии и картографии Российской Федерации \\
        6321 & Заслуженный географ Российской Федерации \\
        6325 & Заслуженный работник дипломатической службы Российской Федерации \\
        6330 & Заслуженный работник жилищно-коммунального хозяйства Российской Федерации \\
        6335 & Заслуженный работник здравоохранения Российской Федерации \\
        6340 & Заслуженный работник культуры Российской Федерации \\
        6350 & Заслуженный работник лесной промышленности Российской Федерации \\
        6360 & Заслуженный работник нефтяной и газовой промышленности Российской Федерации \\
        6370 & Заслуженный работник пищевой индустрии Российской Федерации \\
        6372 & Заслуженный работник прокуратуры Российской Федерации \\
        6374 & Заслуженный работник пожарной охраны Российской Федерации \\
        6375 & Заслуженный работник ракетно-космической промышленности Российской Федерации \\
        6380 & Заслуженный работник рыбного хозяйства Российской Федерации \\
        6381 & Заслуженный ветеринарный врач Российской Федерации \\
        6385 & Заслуженный работник связи и информации Российской Федерации \\
        6390 & Заслуженный работник сельского хозяйства Российской Федерации \\
        6395 & Заслуженный работник социальной защиты населения Российской Федерации \\
        6410 & Заслуженный работник текстильной и легкой промышленности Российской Федерации \\
        6430 & Заслуженный работник транспорта Российской Федерации \\
        6440 & Заслуженный работник физической культуры Российской Федерации \\
        6455 & Заслуженный сотрудник органов внешней разведки Российской Федерации \\
        6457 & Заслуженный сотрудник органов государственной охраны Российской Федерации \\
        6465 & Заслуженный сотрудник органов безопасности Российской Федерации \\
        6469 & Заслуженный сотрудник органов внутренних дел Российской Федерации \\
        6471 & Заслуженный сотрудник следственных органов Российской Федерации \\
        6475 & Заслуженный спасатель Российской Федерации \\
        6480 & Заслуженный строитель Российской Федерации \\
        6484 & Заслуженный судебный пристав Российской Федерации \\
        6485 & Заслуженный таможенник Российской Федерации \\
        6495 & Заслуженный учитель Российской Федерации \\
        6520 & Заслуженный химик Российской Федерации \\
        6530 & Заслуженный художник Российской Федерации \\
        6540 & Заслуженный шахтер Российской Федерации \\
        6542 & Заслуженный штурман Российской Федерации \\
        6545 & Заслуженный штурман-испытатель Российской Федерации \\
        6548 & Заслуженный эколог Российской Федерации \\
        6550 & Заслуженный экономист Российской Федерации \\
        6560 & Заслуженный энергетик Российской Федерации \\
        6570 & Заслуженный юрист Российской Федерации \\
        6590 & Летчик-космонавт Российской Федерации \\
    \end{longtable}
\end{center}

\subsection{ОКИН: 60 --- Награды СССР}

\begin{center}
    \begin{longtable}{rp{.8\textwidth}}
        \hline
        \textbf{Код} & \textbf{Наименование} \\ \hline
        100 & Ордена СССР \\
        101 & Орден Ленина \\
        102 & Орден Октябрьской Революции \\
        103 & Орден "Победа" \\
        104 & Орден Красного Знамени \\
        105 & Орден Суворова I степени \\
        106 & Орден Суворова II степени \\
        107 & Орден Суворова III степени \\
        108 & Орден Ушакова I степепи \\
        109 & Орден Ушакова II пенсии \\
        110 & Орден Кутузова I степени \\
        111 & Орден Кутузова II степени \\
        112 & Орден Кутузова III степени \\
        113 & Орден Нахимова I степени \\
        114 & Орден Нахимова II степени \\
        115 & Орден Богдана Хмельницкого I степени \\
        116 & Орден Богдана Хмельницкого II степени \\
        117 & Орден Богдана Хмельницкого III степени \\
        118 & Орден Александра Невского \\
        119 & Орден Отечественной войны I степени \\
        120 & Орден Отечественной войны II степени \\
        121 & Орден Трудового Красного Знамени \\
        122 & Орден Дружбы народов \\
        123 & Орден Красной Звезды \\
        124 & Орден "За службу Родине в Вооруженных Силах СССР" I степени \\
        125 & Орден "За службу Родине в Вооруженных Силах СССР" II степени \\
        126 & Орден "За службу Родине в Вооруженных Силах СССР" III степени \\
        127 & Орден "Знак Почета" \\
        128 & Орден Славы I степени \\
        129 & Орден Славы II степени \\
        130 & Орден Славы III степени \\
        131 & Орден Трудовой Славы I степени \\
        132 & Орден Трудовой Славы II степени \\
        133 & Орден Трудовой Славы III степени \\
        134 & Орден "Мать-героиня" \\
        135 & Орден "Материнская слава" I степени \\
        136 & Орден "Материнская слава" II степени \\
        137 & Орден "Материнская слава" III степени \\
        138 & Орден Почета \\
        139 & Орден "За личное мужество" \\
        200 & Медали СССР \\
        201 & Медаль "Золотая Звезда" \\
        202 & Золотая медаль "Серп и молот" \\
        203 & Медаль "За отвагу" \\
        204 & Медаль Ушакова \\
        205 & Медаль "За боевые заслуги" \\
        206 & Медаль "За отличие в охране государственной границы СССР" \\
        207 & Медаль "За отличную службу по охране общественного порядка" \\
        208 & Медаль Нахимова \\
        209 & Юбилейная медаль "XX лет РККА" \\
        210 & Медаль "За трудовую доблесть" \\
        211 & Медаль "За трудовое отличие" \\
        212 & Юбилейная медаль "За доблестный труд в ознаменование 100-летия со дня рождения Владимира Ильича Ленина \\
        213 & Юбилейная медаль "За воинскую доблесть в ознаменование 100-летия со дня рождения Владимира Ильича Ленина" \\
        214 & Медаль "За отличие в воинской службе" I степени \\
        215 & Медаль "За отличие в воинской службе" II степени \\
        216 & Медаль "За отвагу на пожаре" \\
        217 & Медаль "За спасение утопающих" \\
        218 & Медаль "Партизану Отечественной воины I степени" \\
        219 & Медаль "Партизану Отечественной войны II степени" \\
        220 & Медаль "За оборону Ленинграда" \\
        221 & Медаль "За оборону Москвы" \\
        222 & Медаль "За оборону Одессы" \\
        223 & Медаль "За оборону Севастополя" \\
        224 & Медаль "За оборону Сталинграда" \\
        225 & Медаль "За оборону Киева" \\
        226 & Медаль "За оборону Кавказа" \\
        227 & Медаль "За оборону Советского Заполярья" \\
        228 & Медаль "За победу над Германией в Великой Отечественной войне 1941 - 1945 гг." \\
        229 & Юбилейная медаль "Двадцать лет Победы в Великой Отечественной войне 1941 - 1945 гг." \\
        230 & Юбилейная медаль "Тридцать лет Победы в Великой Отечественной войне 1941 - 1945 гг." \\
        231 & Медаль "За победу над Японией" \\
        232 & Медаль "За взятие Будапешта" \\
        233 & Медаль "За взятие Кенигсберга" \\
        234 & Медаль "За взятие Вены" \\
        235 & Медаль "За взятие Берлина" \\
        236 & Медаль "За освобождение Белграда" \\
        237 & Медаль "За освобождение Варшавы" \\
        238 & Медаль "За освобождение Праги" \\
        239 & Медаль "За доблестный труд в Великой Отечественной войне 1941 - 1945 гг." \\
        240 & Медаль "Ветеран труда" \\
        241 & Медаль "За восстановление предприятии черной металлургии Юга" \\
        242 & Медаль "За восстановление угольных шахт Донбасса" \\
        243 & Медаль "За освоение целинных земель" \\
        244 & Медаль "За строительство Байкало-Амурской магистрали" \\
        245 & Медаль "В память 800-летия Москвы" \\
        246 & Медаль "В память 250-летия Ленинграда" \\
        247 & Юбилейная медаль "30 лет Советской Армии и Флота" \\
        248 & Юбилейная медаль "40 лет Вооруженных сил СССР" \\
        249 & Юбилейная медаль "50 лет Вооруженных сил СССР" \\
        250 & Юбилейная медаль "50 лет Советской милиции" \\
        251 & Медаль "Медаль материнства" I степени \\
        252 & Медаль "Медаль материнства" II степени \\
        253 & Медаль "За преобразование Нечерноземья РСФСР" \\
        254 & Юбилейная медаль "60 лет Вооруженных сил СССР" \\
        255 & Медаль "За укрепление боевого содружества" \\
        256 & Медаль "В память 1500-летия Киева" \\
        257 & Медаль "Ветеран Вооруженных Сил СССР" \\
        258 & Медаль "За освоение недр и развитие нефтегазового комплекса Западной Сибири" \\
        259 & Юбилейная медаль "40 лет Победы в Великой Отечественной войне 1941 - 1945 гг." \\
        260 & Юбилейная медаль "70 лет Вооруженных сил СССР" \\
    \end{longtable}
\end{center}


\twocolumn[
\subsection{ОКИН: 63 --- Почетные звания СССР}
]
\begin{xtabular}{rp{.38\textwidth}}
    \hline
    \textbf{Код} & \textbf{Наименование} \\ \hline
    100 & Высшие степени отличия и почетные звания СССР \\
    101 & Герой Советского Союза \\
    102 & Герой Социалистического труда \\
    103 & Лауреат Ленинской премии \\
    104 & Лауреат Государственной премии \\
    105 & Мать-героиня \\
    106 & Народный артист СССР \\
    107 & Народный художник СССР \\
    108 & Народный архитектор СССР \\
    109 & Заслуженный летчик-испытатель СССР \\
    110 & Заслуженный штурман-испытатель СССР \\
    111 & Летчик-космонавт СССР \\
    116 & Народный врач СССР \\
    117 & Народный учитель СССР \\
    118 & Заслуженный мастер спорта СССР \\
    119 & Заслуженный тренер СССР \\
    120 & Заслуженный изобретатель СССР \\
    121 & Заслуженный работник сельского хозяйства СССР \\
    127 & Заслуженный работник промышленности СССР \\
    128 & Заслуженный строитель СССР \\
    129 & Заслуженный работник транспорта СССР \\
    130 & Заслуженный работник связи СССР \\
    131 & Заслуженный специалист вооруженных сил СССР \\
    200 & Почетные звания союзных республик \\
    201 & Лауреат республиканской Государственной премии \\
    206 & Заслуженный работник промышленности \\
    208 & Заслуженный энергетик \\
    210 & Заслуженный нефтяник \\
    211 & Мастер нефти \\
    212 & Заслуженный работник газовой промышленности \\
    213 & Заслуженный работник нефтяной и газовой промышленности \\
    216 & Заслуженный шахтер \\
    220 & Заслуженный горняк \\
    222 & Заслуженный металлург \\
    230 & Заслуженный химик \\
    240 & Заслуженный машиностроитель \\
    246 & Заслуженный работник лесной промышленности \\
    256 & Заслуженный работник пищевой индустрии \\
    260 & Заслуженный винодел \\
    264 & Заслуженный работник рыбного хозяйства \\
    265 & Заслуженный рыбак \\
    266 & Заслуженный рыбовод \\
    272 & Заслуженный полиграфист \\
    275 & Заслуженный работник сельского хозяйства \\
    276 & Заслуженный мастер земледелия \\
    277 & Заслуженный агроном \\
    278 & Заслуженный инженер сельского хозяйства \\
    285 & Заслуженный кукурузовод \\
    290 & Мастер хлопка \\
    291 & Заслуженный хлопкороб \\
    300 & Заслуженный садовод \\
    301 & Мастер-садовод \\
    302 & Мастер-виноградарь \\
    304 & Мастер-питомниковод \\
    306 & Мастер чая \\
    310 & Мастер табака \\
    314 & Заслуженный зоотехник \\
    315 & Заслуженный работник животноводства \\
    316 & Заслуженный животновод \\
    317 & Заслуженный мастер животноводства \\
    318 & Мастер животноводства \\
    319 & Заслуженный овцевод \\
    320 & Мастер овцеводства \\
    327 & Заслуженный землеустроитель \\
    328 & Заслуженный механизатор сельского хозяйства \\
    329 & Заслуженный механизатор \\
    330 & Мастер машинной уборки хлеба \\
    332 & Заслуженный ирригатор \\
    333 & Заслуженный мелиоратор \\
    334 & Заслуженный гидротехник \\
    335 & Мастер полива \\
    339 & Заслуженный ветеринарный врач \\
    345 & Заслуженный лесовод \\
    346 & Заслуженный работник охраны природы \\
    350 & Заслуженный работник транспорта \\
    351 & Заслуженный работник автотранспорта \\
    360 & Заслуженный связист \\
    361 & Заслуженный работник связи \\
    370 & Заслуженный строитель \\
    380 & Заслуженный работник торговли и общественного питания \\
    381 & Заслуженный работник торговли \\
    382 & Заслуженный работник торговли и бытового обслуживания \\
    390 & Заслуженный работник бытового обслуживания населения \\
    391 & Заслуженный работник службы быта \\
    392 & Заслуженный работник коммунального хозяйства \\
    393 & Заслуженный работник жилищно-коммунального хозяйства \\
    394 & Заслуженный работник коммунального и бытового обслуживания населения \\
    395 & Заслуженный работник коммунально-бытовой службы \\
    399 & Народный врач \\
    400 & Заслуженный работник здравоохранения \\
    401 & Заслуженный врач \\
    402 & Заслуженный провизор \\
    403 & Заслуженный фармацевт \\
    408 & Заслуженный деятель физкультуры и спорта \\
    409 & Заслуженный деятель спорта \\
    410 & Заслуженный деятель физической культуры \\
    411 & Заслуженный работник физической культуры и спорта \\
    412 & Заслуженный тренер \\
    415 & Заслуженный работник социального обеспечения \\
    420 & Заслуженный учитель школы \\
    421 & Заслуженный учитель \\
    422 & Заслуженный учитель профессионально-технического образования \\
    423 & Заслуженный мастер профессионально-технического образования \\
    424 & Заслуженный работник профессионально-технического образования \\
    425 & Заслуженный преподаватель \\
    426 & Заслуженный работник высшей школы \\
    427 & Заслуженный работник народного образования \\
    428 & Заслуженный деятель высшей школы \\
    431 & Заслуженный пропагандист \\
    435 & Народный артист \\
    436 & Заслуженный артист \\
    437 & Заслуженный деятель искусств \\
    438 & Народный художник \\
    439 & Заслуженный художник \\
    440 & Народный писатель \\
    441 & Заслуженный писатель \\
    442 & Народный поэт \\
    443 & Народный певец \\
    444 & Народный акын \\
    445 & Заслуженный журналист \\
    446 & Заслуженный деятель культуры \\
    447 & Заслуженный работник культурно-просветительской работы \\
    448 & Заслуженный работник культуры \\
    450 & Заслуженный архитектор \\
    452 & Заслуженный библиотекарь \\
    455 & Заслуженная ковровщица \\
    456 & Мастер прикладного искусства \\
    457 & Заслуженный мастер народного творчества \\
    465 & Заслуженный деятель науки и техники \\
    466 & Заслуженный деятель науки \\
    469 & Заслуженный геолог \\
    470 & Заслуженный геолог-разведчик \\
    471 & Заслуженный работник геологии \\
    472 & Заслуженный работник геологической службы \\
    473 & Заслуженный работник геодезии и картографии \\
    482 & Заслуженный юрист \\
    489 & Заслуженный инженер \\
    490 & Заслуженный изобретатель \\
    491 & Заслуженный рационализатор \\
    492 & Заслуженный мастер \\
    493 & Заслуженный экономист \\
    494 & Заслуженный бухгалтер \\
    495 & Заслуженный наставник (работающей, рабочей) молодежи \\
    500 & Почетные звания автономных республик \\
    506 & Заслуженный работник промышленности \\
    508 & Заслуженный энергетик \\
    510 & Заслуженный нефтяник \\
    514 & Заслуженный деятель нефтяной и нефтехимической промышленности \\
    520 & Заслуженный горняк \\
    530 & Заслуженный химик \\
    540 & Заслуженный машиностроитель \\
    552 & Заслуженный работник целлюлозно-бумажной и деревообрабатывающей промышленности \\
    553 & Заслуженный работник лесной промышленности \\
    560 & Заслуженный винодел \\
    567 & Заслуженный работник рыбной промышленности \\
    568 & Заслуженный работник медицинской промышленности \\
    569 & Заслуженный медицинский работник \\
    575 & Заслуженный работник сельского хозяйства \\
    577 & Заслуженный агроном \\
    580 & Заслуженный полевод \\
    584 & Заслуженный рисовод \\
    585 & Заслуженный кукурузовод \\
    590 & Мастер хлопка \\
    591 & Заслуженный хлопкороб \\
    600 & Заслуженный садовод \\
    602 & Мастер-виноградарь \\
    603 & Заслуженный виноградарь \\
    610 & Мастер табака \\
    614 & Заслуженный зоотехник \\
    616 & Заслуженный животновод \\
    624 & Заслуженный шелковод \\
    627 & Заслуженный землеустроитель \\
    628 & Заслуженный механизатор сельского хозяйства \\
    629 & Заслуженный механизатор \\
    630 & Мастер машинной уборки хлеба \\
    632 & Заслуженный ирригатор \\
    633 & Заслуженный мелиоратор \\
    635 & Мастер полива \\
    639 & Заслуженный ветеринарный врач \\
    645 & Заслуженный лесовод \\
    650 & Заслуженный работник транспорта \\
    652 & Заслуженный шофер \\
    653 & Заслуженный водитель \\
    660 & Заслуженный связист \\
    661 & Заслуженный работник связи \\
    670 & Заслуженный строитель \\
    680 & Заслуженный работник торговли и общественного питания \\
    681 & Заслуженный работник торговли \\
    690 & Заслуженный работник бытового обслуживания населения \\
    691 & Заслуженный работник службы быта \\
    693 & Заслуженный работник жилищно-коммунального хозяйства \\
    700 & Заслуженный работник здравоохранения \\
    701 & Заслуженный врач \\
    702 & Заслуженный провизор \\
    708 & Заслуженный деятель физкультуры и спорта \\
    711 & Заслуженный работник физической культуры и спорта \\
    719 & Заслуженный деятель школы \\
    720 & Заслуженный учитель школы \\
    722 & Заслуженный учитель профессионально-технического образования \\
    723 & Заслуженный мастер профессионально-технического образования \\
    724 & Заслуженный работник профессионально-технического образования \\
    726 & Заслуженный работник высшей школы \\
    735 & Народный артист \\
    736 & Заслуженный артист \\
    737 & Заслуженный деятель искусств \\
    738 & Народный художник \\
    739 & Заслуженный художник \\
    740 & Народный писатель \\
    741 & Заслуженный писатель \\
    742 & Народный поэт \\
    745 & Заслуженный журналист \\
    748 & Заслуженный работник культуры \\
    752 & Заслуженный библиотекарь \\
    764 & Заслуженный деятель науки и культуры \\
    765 & Заслуженный деятель науки и техники \\
    766 & Заслуженный деятель науки \\
    769 & Заслуженный геолог \\
    782 & Заслуженный юрист \\
    786 & Заслуженный работник милиции \\
    788 & Заслуженный техник \\
    789 & Заслуженный инженер \\
    790 & Заслуженный изобретатель \\
    791 & Заслуженный рационализатор \\
    793 & Заслуженный экономист \\
    794 & Заслуженный бухгалтер \\
    797 & Заслуженный работник народного хозяйства \\
    798 & Заслуженный рационализатор и изобретатель \\
\end{xtabular}
\onecolumn

\subsection{ОКИН: 66 — Ведомственные награды, установленные федеральными органами исполнительной власти Российской Федерации}

\begin{center}
    \begin{longtable}{rp{.8\textwidth}}
        \hline
        \textbf{Код} & \textbf{Наименование} \\ \hline
        7015 & Звание "Почетный военный топограф" Министерства обороны Российской Федерации \\
        7025 & Звание "Почетный разведчик недр" Министерства природных ресурсов Российской Федерации \\
        7026 & Звание "Почетный работник водного хозяйства" Министерства природных ресурсов Российской Федерации \\
        7030 & Звание "Лучший по профессии на железнодорожном транспорте" Министерства путей сообщения Российской Федерации и Президиума ЦК Российского профсоюза железнодорожников и транспортных строителей \\
        7035 & Звание "Почетный работник топливно-энергетического комплекса" Министерства топлива и энергетики Российской Федерации \\
        7036 & Звание "Заслуженный работник Министерства топлива и энергетики Российской Федерации" \\
        7037 & Звание "Почетный нефтяник" Министерства топлива и энергетики Российской Федерации \\
        7038 & Звание "Почетный нефтехимик" Министерства топлива и энергетики Российской Федерации \\
        7039 & Звание "Почетный работник угольной промышленности" Министерства топлива и энергетики Российской Федерации \\
        7040 & Звание "Почетный шахтер" Министерства топлива и энергетики Российской Федерации \\
        7041 & Звание "Почетный энергетик" Министерства топлива и энергетики Российской Федерации \\
        7042 & Звание "Почетный работник топливно-энергетического комплекса" Министерства энергетики Российской Федерации \\
        7043 & Звание "Почетный работник газовой промышленности" Министерства энергетики Российской Федерации \\
        7044 & Звание "Почетный нефтяник" Министерства энергетики Российской Федерации \\
        7045 & Звание "Почетный нефтехимик" Министерства энергетики Российской Федерации \\
        7046 & Звание "Почетный строитель" Министерства энергетики Российской Федерации \\
        7047 & Звание "Почетный работник угольной промышленности" Министерства энергетики Российской Федерации \\
        7048 & Звание "Почетный шахтер" Министерства энергетики Российской Федерации \\
        7049 & Звание "Почетный энергетик" Министерства энергетики Российской Федерации \\
        7050 & Звание "Почетный горняк" Министерства энергетики Российской Федерации \\
        7065 & Звание "Почетный химик" Министерства экономики Российской Федерации \\
        7066 & Звание "Почетный машиностроитель" Министерства экономики Российской Федерации \\
        7067 & Звание "Почетный работник лесной промышленности" Министерства экономики Российской Федерации \\
        7068 & Звание "Почетный металлург" Министерства экономики Российской Федерации \\
        7070 & Звание "Почетный горняк" Министерства промышленности, науки и технологий Российской Федерации \\
        7071 & Звание "Почетный машиностроитель" Министерства промышленности, науки и технологий Российской Федерации \\
        7072 & Звание "Почетный металлург" Министерства промышленности, науки и технологий Российской Федерации \\
        7073 & Звание "Почетный работник лесной промышленности" Министерства промышленности, науки и технологий Российской Федерации \\
        7074 & Звание "Почетный химик" Министерства промышленности, науки и технологий Российской Федерации \\
        7075 & Звание "Почетный работник текстильной и легкой промышленности" Министерства промышленности, науки и технологий Российской Федерации \\
        7077 & Почетное звание "Почетный метролог" Министерства промышленности и торговли Российской Федерации \\
        7080 & Звание "Лучший рационализатор" Министерства оборонной промышленности Российской Федерации \\
        7081 & Звание "Лучший изобретатель" Министерства оборонной промышленности Российской Федерации \\
        7082 & Звание "Лучший патентовед" Министерства оборонной промышленности Российской Федерации \\
        7091 & Почетное звание «Ветеран сферы воспитания и образования» \\
        7110 & Звание "Почетный авиастроитель" Российского авиационно-космического агентства \\
        7115 & Почетное звание "Почетный сотрудник Росгвардии" \\
        7120 & Знак отличия в труде "Ветеран атомной энергетики и промышленности" Министерства Российской Федерации но атомной энергии \\
        7121 & Знак отличия в труде "Ветеран атомной энергетики и промышленности" Федерального агентства по атомной энергии \\
        7122 & Нагрудный знак "Академик И.В. Курчатов" 1 степени Федерального агентства по атомной энергии \\
        7123 & Нагрудный знак "Академик И.В. Курчатов" 2 степени Федерального агентства по атомной энергии \\
        7124 & Нагрудный знак "Академик И.В. Курчатов" 3 степени Федерального агентства по атомной энергии \\
        7125 & Нагрудный знак "Почетный сотрудник МВД" Министерства внутренних дел Российской Федерации \\
        7128 & Нагрудный знак "Академик П.В. Курчатов" 4 степени Федерального агентства по атомной энергии \\
        7129 & Нагрудный знак "Е.П. Славский" Федерального агентства по атомной энергии \\
        7130 & Нагрудный знак "Почетный знак МЧС России" Министерства Российской Федерации по делам гражданской обороны, чрезвычайным ситуациям и ликвидации последствий стихийных бедствий \\
        7131 & Нагрудный знак "Участнику ликвидации последствий ЧС" Министерства Российской Федерации по делам гражданской обороны, чрезвычайным ситуациям и ликвидации последствий стихийных бедствий \\
        7132 & Нагрудный знак "За заслуги" Министерства Российской Федерации по делам гражданской обороны, чрезвычайным ситуациям и ликвидации последствий стихийных бедствии \\
        7150 & Значок "Почетный радист" Министерства печати и информации Российской Федерации \\
        7155 & Нагрудный знак «Почетный сотрудник Федеральной службы по финансовому мониторингу» \\
        7160 & Нагрудный знак "Отличнику здравоохранения" Министерства здравоохранения Российской Федерации \\
        7170 & Нагрудный знак "За достижения и культуре" Министерства культуры Российской Федерации \\
        7175 & Почетный знак "За заслуги в развитии физической культуры и спорта" Федерального агентства по физической культуре и спорту \\
        7176 & Нагрудный знак "Отличник физической культуры и спорта" Федерального агентства по физической культуре и спорту \\
        7180 & Нагрудный знак "Командир корабля" Министерства обороны Российской Федерации \\
        7181 & Нагрудный знак "Отличник военного строительства" Министерства обороны Российской Федерации \\
        7182 & Нагрудный знак "За службу в полевых учреждениях Банка России" Министерства обороны Российской Федерации \\
        7183 & Знак отличия "За службу в ракетных поисках стратегического назначения" Министерства обороны Российской Федерации \\
        7184 & Знак отличия офицеров Генерального штаба Вооруженных Сил Российской Федерации Министерства обороны Российской Федерации \\
        7185 & Знак отличия "Метрологическая служба Вооруженных Сил Российской Федерации" Министерства обороны Российской Федерации \\
        7186 & Нагрудный знак "За дальний поход" Министерства обороны Российской Федерации \\
        7187 & Нагрудный знак "За разминирование" Министерства обороны Российской Федерации \\
        7188 & Нагрудный знак "Чемпионат Вооруженных Сил России" (I место) Министерства обороны Российской Федерации \\
        7189 & Нагрудный знак "Чемпионат Вооруженных Сил России" (II место) Министерства обороны Российской Федерации \\
        7190 & Нагрудный знак "Чемпионат Вооруженных Сил России" (III место) Министерства обороны Российской Федерации \\
        7191 & Нагрудный знак "Первенство Вооруженных Сил России" (I место) Министерства обороны Российской Федерации \\
        7192 & Нагрудный знак "Первенство Вооруженных Сил России" (II место) Министерства обороны Российской Федерации \\
        7193 & Нагрудный знак "Первенство Вооруженных Сил России" (III место) Министерства обороны Российской Федерации \\
        7194 & Нагрудный знак "Главный маршал артиллерии Неделин" Министерства обороны Российской Федерации \\
        7195 & Нагрудный знак "За службу в военной разведке" Министерства обороны Российской Федерации \\
        7200 & Нагрудный знак «Почётный сотрудник МВД» Министерства внутренних дел Российской Федерации \\
        7201 & Нагрудный знак «За отличие в службе в особых условиях» Министерства внутренних дел Российской Федерации \\
        7202 & Нагрудный знак «За отличную службу в МВД» Министерства внутренних дел Российской Федерации \\
        7203 & Нагрудный знак «Отличник полиции» Министерства внутренних дел Российской Федерации \\
        7204 & Нагрудный знак «За содействие МВД» Министерства внутренних дел Российской Федерации \\
        7205 & Нагрудный знак «Участник боевых действий» Министерства внутренних дел Российской Федерации \\
        7210 & Нагрудный знак "Почетный работник среднего профессионального образования России" Государственного комитета Российской Федерации по высшему образованию \\
        7211 & Нагрудный знак "Почетный работник общего образования Российской Федерации" Министерства образования Российской Федерации \\
        7212 & Нагрудный знак "Почетный работник начального профессионального образования Российской Федерации" Министерства образования Российской Федерации \\
        7213 & Нагрудный знак "Почетный работник среднего профессионального образования Российской Федерации" Министерства образования Российской Федерации \\
        7214 & Нагрудный знак "Почетный работник высшего профессионального образования Российской Федерации" Министерства образования Российской Федерации \\
        7215 & Нагрудный знак "Почетный работник высшего профессионального образования России" Государственного комитета Российской Федерации по высшему образованию \\
        7222 & Нагрудный знак «Почетный работник воспитания и просвещения Российской Федерации» \\
        7230 & Значок "Отличник разведки недр" Министерства природных ресурсов Российской Федерации \\
        7231 & Значок "Отличник водного хозяйства" Министерства природных ресурсов Российской Федерации \\
        7235 & Нагрудный знак "За заслуги в области стандартизации и качества" имени В.В. Бойцова Министерства промышленности и торговли Российской Федерации \\
        7240 & Памятный нагрудный знак "130 лет Министерству путей сообщения" \\
        7241 & Знак "Почетному железнодорожнику" Министерства путей сообщения Российской Федерации \\
        7250 & Нагрудный знак "Шахтерская слава" I степени Министерства топлива и энергетики Российской Федерации \\
        7251 & Нагрудный знак "Шахтерская слава" II степени Министерства топлива и энергетики Российской Федерации \\
        7252 & Нагрудный знак "Шахтерская слава" III степени Министерства топлива и энергетики Российской Федерации \\
        7253 & Нагрудный знак "Трудовая слава" I степени Министерства топлива и энергетики Российской Федерации \\
        7254 & Нагрудный знак "Трудовая слава" II степени Министерства топлива и энергетики Российской Федерации \\
        7255 & Нагрудный знак "Трудовая слава" III степени Министерства топлива и энергетики Российской Федерации \\
        7256 & Нагрудный знак "Шахтерская слава" I степени Министерства энергетики Российской Федерации \\
        7257 & Нагрудный знак "Шахтерская слава" II степени Министерства энергетики Российской Федерации \\
        7258 & Нагрудный знак "Шахтерская слава" II степени Министерства энергетики Российской Федерации \\
        7259 & Нагрудный знак "Трудовая слава" I степени Министерства энергетики Российской Федерации \\
        7260 & Нагрудный знак "Трудовая слава" II степени Министерства энергетики Российской Федерации \\
        7261 & Нагрудный так "Трудовая слава" III степени Министерства энергетики Российской Федерации \\
        7270 & Нагрудный знак "Почетный работник транспорта России" Министерства транспорта Российской Федерации \\
        7271 & Нагрудный знак "Почетный работник Российской транспортной инспекции" Министерства транспорта Российской Федерации \\
        7272 & Нагрудный значок "Почетный автотранспортник" Министерства транспорта Российской Федерации \\
        7273 & Нагрудный значок "Почетному работнику морского флота" Министерства транспорта Российской Федерации \\
        7274 & Нагрудный значок "Почетному полярнику" Министерства транспорта Российской Федерации \\
        7275 & Нагрудный значок "Почетный работник Речного флота" Министерства транспорта Российской Федерации \\
        7276 & Нагрудный значок "Отличник речного флота" Министерства транспорта Российской Федерации \\
        7277 & Нагрудный значок "Почетный работник горэлектротранспорта" Министерства транспорта Российской Федерации \\
        7278 & Знак "За безаварийный налет часов" Министерства транспорта Российской Федерации \\
        7279 & Нагрудный значок "Почетный дорожник" Министерства транспорта Российской Федерации \\
        7280 & Нагрудный значок "100 лет Российскому трамваю" Министерства транспорта Российской Федерации \\
        7281 & Значок "Отличник воздушного транспорта" Министерства транспорта Российской Федерации \\
        7290 & Нагрудный знак "Почетный работник Минтруда России" \\
        7291 & Нагрудный знак "Милосердие" Министерства труда и социальной защиты Российской Федерации \\
        7292 & Нагрудный знак "Отличник социально-трудовой сферы" Министерства труда и социальной защиты Российской Федерации \\
        7300 & Нагрудный знак "Отличник финансовой работы" Министерства финансов Российской Федерации \\
        7305 & Нагрудный знак "Конструктор стрелкового оружия М.Т. Калашников" Министерства экономики Российской Федерации \\
        7310 & Знак отличия "Почетный работник Министерства экономического развития и торговли Российской Федерации" \\
        7320 & Нагрудный знак "Почетный знак Государственного комитета Российской Федерации по делам Севера" \\
        7321 & Нагрудный знак "Заслуженный работник Государственного комитета Российской Федерации по делам Севера" \\
        7322 & Нагрудный знак "За заслуги в гуманной деятельности по защите коренных малочисленных народов Севера" Государственного комитета Российской Федерации по делам Севера \\
        7323 & Знак отличия Министерства Российской Федерации по развитию Дальнего Востока "За заслуги в труде" \\
        7330 & Нагрудный знак "Почетный кинематографист России" Государственного комитета Российской Федерации по кинематографии \\
        7340 & Нагрудный знак "Почетный работник рыбного хозяйства России" Государственного комитета Российской Федерации по рыболовству \\
        7341 & Нагрудный знак "Почетный работник органов рыбоохраны" Министерства сельского хозяйства и продовольствия Российской Федерации \\
        7342 & Нагрудный знак "Почетный рыбовод" Министерства сельского хозяйства и продовольствия Российской Федерации \\
        7350 & Нагрудный знак "За заслуги в стандартизации" Государственного комитета Российской Федерации по стандартизации и метрологии \\
        7355 & Нагрудный знак "Отличник статистики" Государственного комитета Российской Федерации по статистике \\
        7356 & Нагрудный знак "За активное участие во Всероссийской переписи населения 2002 года" Государственного комитета Российской Федерации по статистике \\
        7357 & Нагрудный знак "Отличник статистики" Федеральной службы государственной статистики \\
        7358 & Нагрудный знак "За активное участие во Всероссийской сельскохозяйственной переписи 2006 года" Федеральной службы государственной статистики \\
        7365 & Знак отличия "Почетный работник жилищно-коммунального хозяйства России" Государственного комитета Российской Федерации по строительству и жилищно-коммунальному комплексу \\
        7366 & Знак отличия "Почетный строитель России" Государственного комитета Российской Федерации но строительству и жилищно-коммунальному комплексу \\
        7375 & Нагрудный знак "Почетный таможенник России" Государственного таможенного комитета Российской Федерации \\
        7376 & Нагрудный знак "Отличник таможенной службы" Государственного таможенного комитета Российской Федерации \\
        7390 & Нагрудный знак "Отличник воздушного транспорта" Федеральной службы воздушного транспорта России \\
        7391 & Нагрудный знак "За безаварийный налет часов" Федеральной службы воздушного транспорта России \\
        7415 & Нагрудный знак "Почетный архивист" Федеральной архивной службы России \\
        7420 & Нагрудный знак "Ветеран Железнодорожных войск" Федеральной службы железнодорожных войск Российской Федерации \\
        7421 & Нагрудный знак "Отличник Железнодорожных войск" Федеральной службы железнодорожных войск Российской Федерации \\
        7422 & Нагрудный знак "Лучший специалист по профессии" Федеральной службы железнодорожных войск Российской Федерации \\
        7423 & Нагрудный знак "Лучший специалист Железнодорожных войск" Федеральной службы железнодорожных войск Российской Федерации \\
        7424 & Нагрудный знак "За отличие в службе" Федеральной службы железнодорожных войск Российской Федерации \\
        7425 & Нагрудный знак "За безупречную службу" I степени Федеральной службы железнодорожных войск Российской Федерации \\
        7426 & Нагрудный знак "За безупречную службу" II степени Федеральной службы железнодорожных войск Российской Федерации \\
        7427 & Нагрудный знак "За безупречную службу" III степени Федеральной службы железнодорожных войск Российской Федерации \\
        7440 & Нагрудный знак "Почетный работник гидрометеослужбы России" \\
        7442 & Нагрудный знак "За отличие" Федерального казначейства \\
        7445 & Нагрудный знак "Почетный сотрудник контрразведки" Федеральной службы контрразведки Российской Федерации \\
        7446 & Нагрудный знак "За службу в контрразведке" I степени Федеральной службы контрразведки Российской Федерации \\
        7447 & Нагрудный знак "За службу в контрразведке" II степени Федеральной службы контрразведки Российской Федерации \\
        7448 & Нагрудный знак "За службу в контрразведке" III степени Федеральной службы контрразведки Российской Федерации \\
        7450 & Нагрудный знак "За отличие в службе в особых условиях" федеральной службы войск национальной гвардии Российской Федерации \\
        7451 & Нагрудный знак "За отличие в службе" I степени федеральной службы войск национальной гвардии Российской Федерации \\
        7452 & Нагрудный знак "За отличие в службе" II степени федеральной службы войск национальной гвардии Российской Федерации \\
        7453 & Нагрудный знак "Участник боевых действий" федеральной службы войск национальной гвардии Российской Федерации \\
        7475 & Нагрудный значок "За сбережение и приумножение лесных богатств Российской Федерации" Федеральной службы лесного хозяйства России \\
        7476 & Нагрудный знак "X лет службы в государственной лесной охране Российской Федерации" Федеральной службы лесного хозяйства России \\
        7477 & Нагрудный знак "XX лет службы в государственной лесной охране Российской Федерации" Федеральной службы лесного хозяйства России \\
        7478 & Нагрудный знак "XXX лет службы в государственной лесной охране Российской Федерации" Федеральной службы лесного хозяйства России \\
        7479 & Нагрудный знак "XL лет службы в государственной лесной охране Российской Федерации" Федеральной службы лесного хозяйства России \\
        7485 & Нагрудный знак "Почетный сотрудник налоговой полиции" Федеральной службы налоговой полиции Российской Федерации \\
        7486 & Нагрудный знак "За службу в налоговой полиции" I степени Федеральной службы налоговой полиции Российской Федерации \\
        7487 & Нагрудный знак "За службу в налоговой полиции" II степени Федеральной службы налоговой полиции Российской Федерации \\
        7488 & Нагрудный знак "За службу в налоговой полиции" III степени Федеральной службы налоговой полиции Российской Федерации \\
        7491 & Знак отличия "Почетный работник ФНС России" \\
        7492 & Знак отличия "Отличник ФНС России" \\
        7500 & Нагрудный знак "Почетный сотрудник Федеральной службы охраны Российской Федерации" \\
        7510 & Знак отличия "За заслуги в пограничной службе" I степени Федеральной пограничной службы Российской Федерации \\
        7511 & Знак отличия "За заслуги в пограничном службе" II степени Федеральной пограничной службы Российской Федерации \\
        7512 & Знак отличия "За службу на Кавказе" Федеральной пограничной службы Российской Федерации \\
        7513 & Знак отличия "За службу в Таджикистане" Федеральной пограничной службы Российской Федерации \\
        7514 & Знак отличия "За службу в Заполярье" Федеральной пограничной службы Российской Федерации \\
        7515 & Знак отличия "За службу на Дальнем Востоке" Федеральной пограничной службы Российской Федерации \\
        7516 & Нагрудный знак "Заслуженный пограничник Российской Федерации" Федеральной пограничной службы Российской Федерации \\
        7517 & Нагрудный знак "Отличник погранслужбы" I степени Федеральной пограничной службы Российской Федерации \\
        7518 & Нагрудный знак "Отличник погранслужбы" II степени Федеральной пограничной службы Российской Федерации \\
        7519 & Нагрудный знак "Отличник погранслужбы" III степени Федеральной пограничной службы Российской Федерации \\
        7525 & Знак Циолковского Российского авиационно-космического агентства \\
        7526 & Знак Королева Российского авиационно-космического агентства \\
        7527 & Знак Гагарина Российского авиационно-космического агентства \\
        7528 & Знак "За обеспечение космических стартов" Российского авиационно-космического агентства \\
        7529 & Знак "За содействие космической деятельности" Российского авиационно-космического агентства \\
        7530 & Знак "За международное сотрудничество в области космонавтики" Российского авиационно-космического агентства \\
        7550 & Нагрудный знак "Почетный работник государственного резерва" Российского агентства по государственным резервам \\
        7555 & Нагрудный значок "Почетный дорожник России" Российского дорожного агентства \\
        7560 & Нагрудный знак "Почетный сотрудник федеральных органов правительственной связи и информации" Федерального агентства правительственной связи и информации при Президенте Российской Федерации \\
        7561 & Нагрудный знак "За честь и достоинство в службе Отечеству" Федерального агентства правительственной связи и информации \\
        7570 & Знак "Лучший государственный инспектор Госгортехнадзора России" \\
        7575 & Нагрудный знак "Почетный работник Госатомнадзора России" \\
        7576 & Нагрудный знак "За выслугу лет" I степени Федерального надзора по ядерной и радиационной безопасности \\
        7577 & Нагрудный знак "За выслугу лет" II степени Федерального надзора по ядерной и радиационной безопасности \\
        7578 & Нагрудный знак "За выслугу лет" III степени Федерального надзора по ядерной и радиационной безопасности \\
        7610 & Почетная грамота Министерства Российской Федерации по атомной энергии и Центрального комитета Российского профсоюза работников атомной энергетики и промышленности \\
        7611 & Почетная грамота Министерства Российской Федерации по атомной энергии и Российского профсоюза работников атомной энергетики и промышленности \\
        7612 & Почетно грамота Федерального агентства по атомной энергии \\
        7615 & Почетная грамота Министерства внутренних дел Российской Федерации \\
        7620 & Почетная грамота Министерства Российской Федерации по делам гражданской обороны, чрезвычайным ситуациям и ликвидации последствий стихийных бедствий \\
        7630 & Почетная грамота Министерства Российской Федерации по антимонопольной политике и поддержке предпринимательства \\
        7636 & Почетная грамота федеральной службы войск национальной гвардии Российской Федерации \\
        7637 & Благодарность директора федеральной службы войск национальной гвардии Российской Федерации — главнокомандующего войсками национальной гвардии Российской Федерации \\
        7640 & Почетная грамота Государственного комитета Российской Федерации по печати и Российского комитета профсоюза работников культуры \\
        7641 & Почетная Грамота Федеральной службы России по телевидению и радиовещанию \\
        7642 & Почетный диплом Федеральной службы России по телевидению и радиовещанию \\
        7650 & Почетная грамота Министерства Российской Федерации по налогам и сборам \\
        7651 & Почетная грамота Федеральной налоговой службы \\
        7652 & Благодарность руководителя Федеральной налоговой службы \\
        7655 & Почетная грамота Министерства здравоохранения Российской Федерации \\
        7660 & Почетная грамота Министерства иностранных дел Российской Федерации \\
        7670 & Почетная грамота Министерства науки и технологий Российской Федерации \\
        7672 & Почетная грамота Министерства промышленности, науки и технологий Российской Федерации \\
        7675 & Почетная грамота Министерства образования Российской Федерации \\
        7676 & Почетная грамота Государственного комитета Российской Федерации по высшему образованию \\
        7679 & Почетная грамота Министерства просвещения Российской Федерации \\
        7680 & Почетная грамота Министерства природных ресурсов Российской Федерации \\
        7683 & Почетная грамота Министерства путей сообщения Российской Федерации и Центрального комитета Российского профсоюза железнодорожников и транспортных строителей \\
        7684 & Благодарность Министра путей сообщения Российской Федерации \\
        7685 & Почетная грамота Министерства Российской Федерации по развитию Дальнего Востока \\
        7686 & Благодарность Министерства Российской Федерации по развитию Дальнего Востока \\
        7690 & Почетная грамота Министерства топлива и энергетики Российской Федерации \\
        7691 & Почетная грамота Министерства энергетики Российской Федерации \\
        7692 & Благодарность Министерства энергетики Российской Федерации \\
        7700 & Почетная грамота Министерства транспорта Российской Федерации \\
        7705 & Почетная грамота Министерства труда и социального развития Российской Федерации \\
        7706 & Почетная грамота Министерства труда и социальной защиты Российской Федерации \\
        7708 & Почетная грамота Министерства финансов Российской Федерации \\
        7709 & Благодарность Министра финансов Российской Федерации \\
        7710 & Почетная грамота Министерства экономики Российской Федерации \\
        7711 & Почетная грамота Комитета Российской Федерации по машиностроению \\
        7712 & Почетная грамота Министерства промышленности Российской Федерации \\
        7713 & Почетная грамота Государственного Комитета Российской Федерации по промышленной политике \\
        7714 & Почетная грамота Комитета Российской Федерации по металлургии и ЦК профсоюза рабочих горно-металлургической промышленности \\
        7715 & Почетная грамота Министерства экономического развития и торговли Российской Федерации \\
        7720 & Почетная грамот Министерства юстиции Российской Федерации \\
        7725 & Почетная грамота "За большой вклад в дело развития Севера" Государственного комитета Российской Федерации по делам Севера \\
        7726 & Диплом "За плодотворный труд в развитии Севера" Государственного комитета Российской Федерации по делам Севера \\
        7735 & Почетная грамота Государственного комитета Российской Федерации по земельной политике \\
        7740 & Почетная грамота Государственного комитета Российской Федерации по охране окружающей среды \\
        7745 & Почетная грамота Государственного комитета Российской Федерации по рыболовству \\
        7755 & Почетная грамота Государственного комитета Российской Федерации по стандартизации и метрологии \\
        7757 & Почетная грамота Федерального агентства по техническому регулированию и метрологии \\
        7760 & Почетная грамота Государственного комитета Российской Федерации по статистике \\
        7761 & Благодарственное письмо Государственного комитета Российской Федерации по статистике \\
        7762 & Почетная грамота Федеральной службы государственной статистики \\
        7763 & Благодарственное письмо Федеральной службы государственной статистики \\
        7765 & Почетная грамота Министерства строительства Российской Федерации \\
        7770 & Почетная грамота Государственного таможенного комитета Российской Федерации \\
        7790 & Почетная грамота Федеральной энергетической комиссии Российской Федерации \\
        7795 & Почетная грамота Федеральной службы воздушного транспорта России \\
        7800 & Почетная грамота Федеральной архивной службы России \\
        7801 & Почетная грамота Государственной архивной службы России \\
        7802 & Почетная грамота Комитета по делам архивов при Правительстве Российской Федерации \\
        7805 & Почетная грамота Федерального казначейства \\
        7806 & Благодарность Федерального казначейства \\
        7810 & Почетная грамота Федерального агентства по физической культуре и спорту \\
        7820 & Почетная грамота Федеральной службы России по гидрометеорологии и мониторингу окружающей среды \\
        7825 & Грамота Федеральной службы безопасности Российской Федерации \\
        7830 & Почетная Грамота "За особый вклад в осуществление валютного и экспортного контроля" Федеральной службы России по валютному и экспортному контролю \\
        7835 & Почетная грамота Федеральной службы России по финансовому оздоровлению и банкротству \\
        7845 & Грамота Федеральной службы охраны Российской Федерации \\
        7846 & Почетная грамота Федеральной службы по финансовому мониторингу \\
        7847 & Почетная грамота территориального органа Федеральной службы по финансовому мониторингу \\
        7848 & Благодарность директора Федеральной службы по финансовому мониторингу \\
        7849 & Благодарность руководителя территориального органа Федеральной службы по финансовому мониторингу \\
        7850 & Диплом I степени Федеральной пограничной службы Российской Федерации \\
        7851 & Диплом II степени Федеральной пограничной службы Российской Федерации \\
        7852 & Диплом III степени Федеральной пограничной службы Российской Федерации \\
        7853 & Занесение фотопортрета на Доску почета Федеральной службы по финансовому мониторингу \\
        7854 & Занесение фотопортрета на Доску почета территориального органа Федеральной службы по финансовому мониторингу \\
        7855 & Почетная грамота Российского авиационно-космического агентства \\
        7858 & Почетная грамота Российского агентства по боеприпасам \\
        7862 & Почетная грамота Российского агентства по обычным вооружениям \\
        7867 & Почетная грамота Российского агентства по системам управления \\
        7875 & Почетная грамота Российского агентства по государственным резервам \\
        7880 & Почетная грамота Федерального агентства правительственной связи и информации при Президенте Российской Федерации \\
        7881 & Грамота Федерального агентства правительственной связи и информации при Президенте Российской Федерации \\
        7885 & Почетная грамота Федерального агентства по делам национальностей \\
        7886 & Благодарность руководителя Федерального агентства по делам национальностей \\
        7887 & Благодарность Министерства просвещения Российской Федерации \\
        7890 & Почетная грамота Федерального горного и промышленного надзора России \\
        7902 & Медаль "За трудовую доблесть" Министерства обороны Российской Федерации \\
        7903 & Медаль "Михаил Калашников" Министерства обороны Российской Федерации \\
        7904 & Медаль "За боевое отличие" федеральной службы войск национальной гвардии Российской Федерации \\
        7905 & Медаль "За проявленную доблесть" I степени федеральной службы войск национальной гвардии Российской Федерации \\
        7906 & Медаль "За проявленную доблесть" II степени федеральной службы войск национальной гвардии Российской Федерации \\
        7907 & Медаль "За проявленную доблесть" III степени федеральной службы войск национальной гвардии Российской Федерации \\
        7908 & Медаль "За разминирование" федеральной службы войск национальной гвардии Российской Федерации \\
        7909 & Медаль "За спасение" федеральной службы войск национальной гвардии Российской Федерации \\
        7910 & Медаль "Ветеран службы" федеральной службы войск национальной гвардии Российской Федерации \\
        7911 & Медаль "За боевое содружество" федеральной службы войск национальной гвардии Российской Федерации \\
        7912 & Медаль "За заслуги в укреплении правопорядка" федеральной службы войск национальной гвардии Российской Федерации \\
        7913 & Медаль "За заслуги в труде" федеральной службы войск национальной гвардии Российской Федерации \\
        7914 & Медаль "За отличие в службе" I степени федеральной службы войск национальной гвардии Российской Федерации \\
        7915 & Медаль "За отличие в службе" II степени федеральной службы войск национальной гвардии Российской Федерации \\
        7916 & Медаль "За отличие в службе" III степени федеральной службы войск национальной гвардии Российской Федерации \\
        7917 & Медаль "За особые достижения в учебе" федеральной службы войск национальной гвардии Российской Федерации \\
        7918 & Медаль "За содействие" федеральной службы войск национальной гвардии Российской Федерации \\
        7919 & Медаль "Генерал армии Яковлев" федеральной службы войск национальной гвардии Российской Федерации \\
        7920 & Медаль "За отличие в военной службе" I степени Министерства Российской Федерации по делам гражданской обороны, чрезвычайным ситуациям и ликвидации последствий стихийных бедствий \\
        7921 & Медаль "За отличие в военной службе" II степени Министерства Российской Федерации по делам гражданской обороны, чрезвычайным ситуациям и ликвидации последствий стихийных бедствий \\
        7922 & Медаль "За отличие в военной службе" III степени Министерства Российской Федерации по делам гражданской обороны, чрезвычайным ситуациям и ликвидации последствий стихийных бедствий \\
        7923 & Медаль "За боевые отличия" Министерства обороны Российской Федерации \\
        7924 & Медаль "За воинскую доблесть" Министерства обороны Российской Федерации \\
        7925 & Медаль «За доблесть в службе» Министерства внутренних дел Российской Федерации \\
        7926 & Медаль «За трудовую доблесть» Министерства внутренних дел Российской Федерации \\
        7927 & Медаль «За безупречную службу в МВД» Министерства внутренних дел Российской Федерации \\
        7928 & Медаль «За отличие в службе» I степени Министерства внутренних дел Российской Федерации \\
        7929 & Медаль «За отличие в службе» II степени Министерства внутренних дел Российской Федерации \\
        7930 & Медаль «И.Д. Путилина» Министерства внутренних дел Российской Федерации \\
        7931 & Медаль «За разминирование» Министерства внутренних дел Российской Федерации \\
        7932 & Медаль «За смелость во имя спасения» Министерства внутренних дел Российской Федерации \\
        7933 & Медаль «За заслуги в службе в особых условиях» Министерства внутренних дел Российской Федерации \\
        7934 & Медаль "За разминирование" Министерства обороны Российской Федерации \\
        7935 & "Памятная медаль А.М. Горчакова" Министерства иностранных дел Российской Федерации \\
        7936 & Медаль "Участнику военной операции в Сирии" Министерства обороны Российской Федерации \\ \\
        7940 & Медаль "За укрепление боевого содружества" Министерства обороны Российской Федерации \\
        7941 & Медаль "За отличие в военной службе" Министерства обороны Российской Федерации \\
        7942 & "Медаль К.Д. Ушинского" Министерства образования Российской Федерации \\
        7944 & Медаль Л.С. Выготского \\
        7945 & "Памятная медаль имени А.А. Бетанкура" Министерства путей сообщения Российской Федерации \\
        7946 & Памятная медаль имени П.П. Мельникова" Министерства путей сообщения Российской Федерации \\
        7948 & "Медаль Петра Лесгафта" Федерального агентства по физическом культуре и спорту \\
        7949 & "Медаль Николая Озерова" Федерального агентства по физической культуре и спорту \\
        7950 & Медаль "За отличие в военной службе" I степени Федеральной службы безопасности Российской Федерации \\
        7951 & Медаль "За отличие в военной службе" II степени Федеральной службы безопасности Российской Федерации \\
        7952 & Медаль "За отличие в военной службе" III степени Федеральной службы безопасности Российской Федерации \\
        7962 & Медаль "20 лет Федеральному казначейству" \\
        7965 & Медаль "За отличие в военной службе" I степени Федеральной службы охраны Российской Федерации \\
        7966 & Медаль "За отличие в военной службе" II степени Федеральной службы охраны Российской Федерации \\
        7967 & Медаль "За отличие в военной службе" III степени Федеральной службы охраны Российской Федерации \\
        7970 & Медаль "За укрепление боевого содружества" Федеральной пограничной службы Российской Федерации \\
        7971 & Медаль "За отличие в военной службе" I степени Федеральной пограничной службы Российской Федерации \\
        7972 & Медаль "За отличие в военной службе" II степени Федеральной пограничной службы Российской Федерации \\
        7973 & Медаль "За отличие в военной службе" III степени Федеральной пограничной службы Российской Федерации \\
        7974 & Медаль "Федеральная пограничная служба Российской Федерации" \\ \\
        7980 & Медаль «За заслуги перед Федеральной службой по финансовому мониторингу» \\
        7981 & Медаль «За безупречную службу» Федеральной службы по финансовому мониторингу I степени \\
        7982 & Медаль «За безупречную службу» Федеральной службы по финансовому мониторингу II степени \\
        7983 & Медаль «За безупречную службу» Федеральной службы по финансовому мониторингу III степени \\
        7984 & Золотая медаль «За содействие» Федеральной службы по финансовому мониторингу \\
        7985 & Серебряная медаль «За содействие» Федеральной службы по финансовому мониторингу \\
        7986 & Медаль «За боевое содружество» Министерства внутренних дел Российской Федерации \\
        7987 & Медаль «За укрепление международного полицейского сотрудничества» Министерства внутренних дел Российской Федераци \\
        7988 & Медаль «За вклад в укрепление правопорядка» Министерства внутренних дел Российской Федерации \\
        7989 & Медаль "За укрепление боевого содружества" Федерального агентства правительственной связи и информации при Президете Российской Федерации \\
        7990 & Медаль "За отличие в военной службе" I степени Федерального агентства правительственной связи и информации при Президенте Российской Федерации \\
        7991 & Медаль "За отличие в военной службе" II степени Федерального агентства правительственной связи и информации при Президенте Российской Федерации \\
        7992 & Медаль "За отличие в военной службе" III степени Федерального агентства правительственной связи и информации при Президенте Российской Федерации \\
        7997 & Медаль "За заслуги в проведении Всероссийской сельскохозяйственной переписи 2006 года" Федеральной службы государственной статистики \\
        7998 & Медаль "Трудовая доблесть" Министерства промышленности и торговли Российской Федерации \\
        7999 & Медаль "За заслуги перед отечественным здравоохранением" \\
    \end{longtable}
\end{center}

\subsection{ОКИН: 67 --- Ведомственные награды в СССР}

\begin{center}
    \begin{longtable}{rp{.8\textwidth}}
        \hline
        \textbf{Код} & \textbf{Наименование} \\ \hline
        8010 & Почетное звание "Почетный энергетик СССР" Министерства энергетики и электрификации СССР \\
        8011 & Почетное звание "Лучший наставник молодежи Минэнерго СССР" \\
        8012 & Почетное звание "Почетный мастер Минэнерго СССР" \\
        8013 & Почетное звание "Почетный нефтяник" Министерства нефтяной промышленности СССР \\
        8014 & Звание "Почетный мастер нефтяной промышленности" Министерства нефтяной промышленности СССР \\
        8015 & Почетное звание "Почетный шахтер" Министерства угольной промышленности СССР \\
        8016 & Почетное звание "Почетный работник угольной промышленности" Министерства угольной промышленности СССР \\
        8017 & Звание "Почетный работник Министерства строительства предприятий нефтяной и газовой промышленности СССР" \\
        8018 & Звание "Отличник Миннефтегазстроя" \\
        8019 & Звание "Почетный мастер газовой промышленности" Министерства газовой промышленности СССР \\
        8020 & Звание "Почетный работник газовой промышленности" Министерства газовой промышленности СССР \\
        8025 & Звание "Мастер связи" Министерства связи СССР \\
        8028 & Звание "Почетный геодезист" Главного управления геодезии и картографии при Совете Министров СССР и Президиума ЦК профсоюза рабочих геологоразведочных работ \\
        8030 & Звание "Почетный мастер Минэнергомаша СССР" \\
        8031 & Звание "Почетный мастер нефтехимической промышленности СССР" \\
        8032 & Звание "Почетный мастер химической и нефтеперерабатывающей промышленности СССР" \\
        8033 & Звание "Почетный мастер химической промышленности СССР" \\
        8034 & Звание "Почетный металлург" \\
        8038 & Звание "Почетный работник Минэнергомаша СССР" \\
        8040 & Звание "Почетный строитель Нечерноземагропромстроя" \\
        8043 & Звание "Почетный полярник" \\
        8044 & Ветеран горноспасательной службы Министерства цветной металлургии СССР \\
        8050 & Звание "Специалист высшего класса" Министерства речного флота РСФСР \\
        8055 & Звание "Лучший технолог" Министерства оборонной промышленности СССР \\
        8060 & Звание "Лучший конструктор" Министерства оборонной промышленности СССР \\
        8061 & Звание "Лучший рационализатор" Министерства оборонной промышленности СССР \\
        8062 & Звание "Лучший изобретатель" Министерства оборонной промышленности СССР \\
        8063 & Звание "Лучший строитель" Министерства оборонной промышленности СССР \\
        8064 & Звание "Лучший инструментальщик" Министерства оборонной промышленности СССР \\
        8065 & Звание "Лучший по профессии" Министерства оборонной промышленности СССР \\
        8066 & Звание "Лучший рабочий" Министерства оборонной промышленности СССР \\
        8067 & Звание "Лучший организатор изобретательской работы" Министерства оборонной промышленности СССР \\
        8068 & Звание "Лучший производственный мастер" Министерства оборонной промышленности СССР \\
        8069 & Звание "Почетный мастер" Министерства оборонной промышленности СССР \\       
        8115 & Нагрудный знак "Почетный сотрудник госбезопасности" Комитета государственной безопасности при Совете Министров СССР \\
        8116 & Юбилейный нагрудный знак "50 лет ВЧК-КГБ" Комитета государственной безопасности при Совете Министров СССР \\
        8117 & Юбилейный нагрудный знак "60 лет ВЧК-КГБ" Комитета государственной безопасности при Совете Министров СССР \\
        8118 & Юбилейный нагрудный знак "70 лет ВЧК-КГБ" Комитета государственной безопасности СССР \\
        8143 & Знак "За ударный труд" Минэнергомаша СССР \\
        8144 & Знак "Отличный путеец" Министерства путей сообщения СССР \\
        8146 & Значок "За работу без аварий" Всероссийского объединения "Россельхозтехника" Совета Министров РСФСР и Республиканского комитета профсоюза работников сельского хозяйства РСФСР \\
        8147 & Значок "Отличник печати" Государственного комитета СССР по делам издательств, полиграфии и книжной торговли и ЦК профсоюза работников культуры \\
        8148 & Значок "За отличную работу" Министерства культуры СССР \\
        8155 & Знак "Лучший инспектор Госгортехнадзора СССР" \\
        8156 & Нагрудный знак "Знак отличия МВТ СССР" \\
        8157 & Нагрудный знак "Знак отличия ГКЭС СССР" \\
        8158 & Нагрудный знак "Знак отличия МВЭС СССР" \\
        8160 & Нагрудный знак "За заслуги в стандартизации" Государственного комитета СССР по стандартизации и метрологии и ЦК профсоюза работников машиностроения и приборостроения \\
        8165 & Значок "Почетный радист" Совета Министров СССР \\
        8168 & Нагрудный знак "За безаварийную работу" Министерства сельского хозяйства и продовольствия СССР \\
        8170 & Нагрудный значок "Почетный автотранспортник" \\
        8171 & Нагрудный значок "Почетному работнику морского флота" \\
        8172 & Нагрудный значок "Почетный работник речного флота" \\
        8173 & Нагрудный значок "Почетному полярнику" \\
        8174 & Нагрудный значок "За работу без аварий" I степени для работников автомобильного транспорта \\
        8175 & Нагрудный значок "За работу без аварий" II степени для работников автомобильного транспорта \\
        8176 & Нагрудный значок "За работу без аварий" III степени для работников автомобильного транспорта \\
        8177 & Нагрудный значок "Отличник речного флота" \\
        8178 & Нагрудный значок "Отличник Министерства автотранспорта РСФСР" \\
        8179 & Нагрудный значок "Отличник социалистического соревнования Министерства морского и речного флота" \\
        8180 & Нагрудный значок "Отличник социалистического соревнования речного флота" \\
        8181 & Лучший изобретатель железнодорожного транспорта \\
        8182 & Почетная грамота Министерства строительства предприятий нефтяной и газовой промышленности СССР \\
        8190 & Лучший организатор технического творчества на железнодорожном транспорте \\
        8191 & Лучший рационализатор железнодорожного транспорта \\
        8192 & Лучший изобретатель легкой промышленности СССР \\
        8194 & Нагрудный знак "За налет" Главного управления Гражданского воздушного Флота Союза ССР \\
        8195 & Нагрудный знак "За механизацию и автоматизацию в металлургии" Министерства черной металлургии СССР и президиума ЦК профсоюза рабочих металлургической промышленности \\
        8196 & Нагрудный знак Н.И. Пирогова "За заслуги в гуманной деятельности" Союза обществ Красного Креста и Красного Полумесяца СССР \\
        8197 & Нагрудный знак "За заслуги в развитии туризма и экскурсий в РСФСР" Российского республиканского совета по туризму и экскурсиям \\
        8201 & Нагрудный значок "Отличник службы быта" Главного управления бытового обслуживания населения при Совете Министров РСФСР и Президиума ЦК профсоюза рабочих коммунально-бытовых предприятий \\
        8202 & Нагрудный значок "Отличник ВОС" \\
        8203 & Нагрудный значок "Отличник советской торговли" Министерства торговли РСФСР \\
        8210 & Нагрудный значок "За успехи в работе" Министерства юстиции Союза СССР \\
        8215 & Нагрудный значок "Отличник социалистического учета" Совета Народных Комиссаров СССР \\
        8220 & Нагрудный значок "Отличник статистики" Государственного комитета СССР по статистике и Президиума ЦК профсоюза работников государственных учреждений \\
        8225 & Нагрудный значок "Отличник бытового обслуживания населения" Росбытсоюза и ЦК профсоюза рабочих местной промышленности и коммунально-бытовых предприятий Российской Федерации \\
        8226 & Нагрудный значок "Отличник Стройбанка СССР" Стройбанка СССР и ЦК профсоюза работников государственных учреждений \\
        8250 & Нагрудный значок "Отличник химической и нефтехимической промышленности СССР" \\
        8260 & Нагрудный значок "Отличник геодезии и картографии" Главного управления геодезии и картографии при Совете Министров СССР и Президиума ЦК профсоюза рабочих геологоразведочных работ \\
        8265 & Нагрудный значок "За сбережение и приумножение лесных богатств РСФСР" Совета Министров РСФСР \\
        8266 & Нагрудный значок "X лет службы в государственной лесной охране СССР" Государственного комитета лесного хозяйства Совета Министров СССР \\
        8267 & Нагрудный значок "XX лет службы в государственной лесной охране СССР" Государственного комитета лесного хозяйства Совета Министров СССР \\
        8268 & Нагрудный значок "XXX лет службы в государственной лесной охране СССР" Государственного комитета лесного хозяйства Совета Министров СССР \\
        8320 & Нагрудный знак "Отличник энергетики и электрификации СССР" Министерства энергетики и электрификации СССР \\
        8321 & Нагрудный значок "50 лет первого Всесоюзного слета стахановцев-энергетиков" Министерства энергетики и электрификации СССР \\
        8322 & Нагрудный значок "50 лет ГОЭЛРО" Министерства энергетики и электрификации СССР \\
        8323 & Нагрудный значок "60 лет ГОЭЛРО" Министерства энергетики и электрификации СССР \\
        8324 & Нагрудный значок "70 лет ГОЭЛРО" Министерства энергетики и электрификации СССР \\
        8325 & Нагрудный значок "Триллион киловатт-часов" Министерства энергетики и электрификации СССР \\
        8326 & Нагрудный значок "50 лет теплофикации СССР" Министерства энергетики и электрификации СССР \\
        8327 & Нагрудный значок "Отличник нефтедобывающей промышленности" Министерства нефтяной промышленности СССР \\
        8328 & Нагрудный значок "Отличник нефтяной промышленности" Министерства нефтяной промышленности СССР \\
        8329 & Нагрудный знак "Шахтерская слава" I степени Министерства угольной промышленности СССР \\
        8330 & Нагрудный знак "Шахтерская слава" II степени Министерства угольной промышленности СССР \\
        8331 & Нагрудный знак "Шахтерская слава" III степени Министерства угольной промышленности СССР \\
        8331 & Нагрудный знак "Трудовая слава" III степени Министерства угольной промышленности СССР \\
        8332 & Нагрудный знак "Трудовая слава" I степени Министерства угольной промышленности СССР \\
        8333 & Нагрудный знак "Трудовая слава" II степени Министерства угольной промышленности СССР \\
        8335 & Нагрудный значок "Отличник Министерства газовой промышленности" \\
        8336 & Нагрудный значок "За развитие энергетики Приангарья" Министерства энергетики и электрификации СССР \\
        8339 & Нагрудный знак "Ветеран спорта СССР" \\
        8340 & Нагрудный знак "За заслуги в развитии физической культуры и спорта" \\
        8341 & Нагрудный знак "Отличник физической культуры и спорта" \\
        8342 & Почетный знак ДОСААФ СССР \\
        8344 & Почетный Знак Советского Красного Креста \\
        8345 & Почетный знак Союза Общества Красного Креста и Красного Полумесяца \\
        8346 & Почетный знак "За вклад в дело дружбы" Союза Советских обществ дружбы и культурной связи с зарубежными странами \\
        8347 & Значок "Отличник качества" Министерства машиностроения СССР \\
        8360 & Нагрудный значок "Отличник просвещения СССР" Министерства Просвещения СССР и Президиума ЦК профсоюза работников просвещения, высшей школы и научных учреждений \\
        8361 & Нагрудный значок "Отличник профессионально-технического образования РСФСР" Государственного комитета СМ РСФСР по профтехобразованию и Президиума ЦК профсоюза работников государственных учреждений \\
        8365 & Нагрудный значок "Отличник социалистического соревнования Министерства связи СССР" \\
        8366 & Знак специальной связи "Ветеран службы" Министерства связи СССР \\
        8368 & Нагрудный значок "Отличник финансовой работы" Министерства финансов СССР и Центрального комитета профсоюза работников госучреждений \\
        8370 & Нагрудный значок "Отличник социалистического соревнования" Министерства оборонной промышленности СССР \\
        8371 & Нагрудный значок "Отличник качества" Министерства оборонной промышленности СССР \\
        8372 & Нагрудный значок "Почетный мастер" Министерства оборонной промышленности СССР \\
        8410 & Почетная грамота Центрального статистического управления при Совете Министров СССР и ЦК профсоюза работников государственных учреждений \\
        8411 & Почетная грамота Центрального статистического управления СССР и ЦК профсоюза работников государственных учреждений \\
        8412 & Почетная грамота Государственного комитета СССР по статистике и ЦК профсоюза работников государственных учреждений \\
        8413 & Почетная грамота Центрального статистического управления при Совете Министров РСФСР и ЦК профсоюза работников государственных учреждений \\
        8414 & Почетная грамота Центрального статистического управления РСФСР и ЦК профсоюза работников государственных учреждений \\
        8415 & Почетная грамота Государственного комитета РСФСР по статистике и ЦК профсоюза работников государственных учреждений \\
        8418 & Грамота "За активное участие в деятельности организаций Обществ Красного Креста СССР" Исполкома Союза обществ Красного Креста и Красного Полумесяца СССР \\
        8423 & Диплом "За активное участие в укреплении мира и дружбы между народами" Российского Комитета защиты мира \\
        8425 & Диплом "За высокие показатели в культурном обслуживании Вооруженных Сил СССР" Президиума Правления Всероссийского театрального общества и Главного политического управления Советской Армии и Военно-морского Флота \\
        8430 & Почетная грамота Госстандарта СССР и ЦК профсоюза работников машиностроения и приборостроения \\
        8435 & Похвальная грамота Главного управления Северного Морского Пути при СНК СССР \\
        8436 & Почетная грамота Министерства морского флота СССР \\
        8437 & Почетная грамота Министерства речного флота РСФСР \\
        8438 & Почетная грамота Министерства автомобильного транспорта РСФСР \\
        8440 & Почетная грамота Главного архивного управления при Совете Министров СССР и ЦК профсоюза работников государственных учреждении \\
        8441 & Почетная грамота Главного архивного управления при Совете Министров РСФСР и ЦК профсоюза работников государственных учреждений \\
        8442 & Почетная грамота Главного архивного управления при Совете Министров РСФСР \\
        8443 & Почетная грамота Главного управления геодезии и картографии при Совете Министров СССР и ЦК профсоюза рабочих геологоразведочных работ \\
        8452 & Почетная грамота "За активную работу в органах народного контроля СССР" Комитета народного контроля СССР \\
        8480 & Почетная грамота Министерства нефтяной промышленности СССР и ЦК профсоюза рабочих нефтяной и газовой промышленности \\
        8481 & Почетная грамота Министерства строительства предприятий нефтяной и газовой промышленности СССР и ЦК профсоюза рабочих нефтяной и газовой промышленности \\
        8483 & Книга Почета Министерства строительства предприятий нефтяной и газовой промышленности СССР с вручением свидетельства установленного образца \\
        8485 & Доска Почета Министерства строительства предприятий нефтяной и газовой промышленности СССР с вручением свидетельства установленного образца \\
        8486 & Почетная грамота Министерства топливной промышленности РСФСР и ЦК профсоюза рабочих электростанций и электротехнической промышленности \\
        8487 & Почетная грамота Министерства топливной промышленности РСФСР и ЦК профсоюза рабочих угольной промышленности \\
        8488 & Почетная грамота Министерства топливной промышленности РСФСР и ЦК профсоюза рабочих лесной, бумажной и деревообрабатывающей промышленности \\
        8489 & Почетная грамота Министерства газовой промышленности СССР и ЦК профсоюза рабочих нефтяной и газовой промышленности \\
        8490 & Занесение в Книгу почета Министерства газовой промышленности СССР и ЦК профсоюза рабочих нефтяной и газовой промышленности \\
        8510 & Грамота КГБ СССР \\
        8511 & Почетная грамота КГБ СССР \\
        8520 & Почетная грамота Министерства просвещения СССР и ЦК профсоюза работников просвещения, высшей школы и научных учреждений \\
        8525 & Почетная грамота Министерства химической и нефтеперерабатывающей промышленности СССР и ЦК профсоюза рабочих химической и нефтехимической промышленности \\
        8526 & Почетная грамота Министерства черной металлургии СССР и ЦК профсоюза рабочих металлургической промышленности \\
        8532 & Почетная грамота Президиума Академии наук СССР в связи с 250-летием Академии наук СССР \\
        8534 & Почетная грамота "За долголетний добросовестный труд по оздоровлению инвалидов по зрению" Президиума Центрального управления Всероссийского общества слепых \\
        8540 & Почетная грамота "За активное участие в деятельности Комитета и в связи с 25-летием Всемирного движения сторонников мира" Советского Комитета защиты мира \\
        8545 & Почетная грамота Министерства финансов СССР и Центрального комитета профсоюза работников госучреждений \\
        8546 & Почетная грамота Министерства финансов РСФСР и Центрального комитета профсоюза работников госучреждений \\
        8552 & Почетная грамота Всероссийского общества "Знание" \\
        8553 & Почетная грамота "За длительную безупречную работу в профсоюзах" ВЦСПС и ФНПР \\
        8560 & Почетная грамота Министерства оборонной промышленности СССР и Президиума ЦК профсоюза авиационной и оборонной промышленности \\
        8580 & "Медаль Н.К. Крупской" Министерства просвещения СССР и Президиума ЦК профсоюза работников просвещения. высшей школы и научных учреждений \\
        8584 & Памятная настольная медаль "50 лет ВЧК-КГБ" Комитета государственной безопасности при Совете Министров СССР \\
        8584 & Памятная настольная медаль "60 лет ВЧК-КГБ" Комитета государственной безопасности при Совете Министров СССР \\
        8585 & Памятная настольная медаль "70 лет ВЧК-КГБ" Комитета государственной безопасности СССР \\
    \end{longtable}
\end{center}

\subsection{ОКИН: 89 --- Занятое население по условиям трудового договора}

\begin{center}
    \begin{tabular}{rp{.8\textwidth}}
        \hline
        \textbf{Код} & \textbf{Наименование} \\ \hline
        01 & На основе трудового договора (служебного контракта) на неопределенный срок \\
        02 & На основе трудового договора (служебного контракта) на определенный срок \\
        03 & На основе трудового договора о выполнении работы на дому (надомника) \\
        04 & На основе трудового договора о выполнении дистанционной работы \\
        05 & На основе договора гражданско-правового характера \\
    \end{tabular}
\end{center}

\subsection{ОКИН: 90 --- Работающие не по найму}

\begin{center}
    \begin{tabular}{rp{.8\textwidth}}
        \hline
        \textbf{Код} & \textbf{Наименование} \\ \hline
        01 & Работодатели \\
        02 & Самостоятельно занятые \\
        03 & Члены производственных кооперативов \\
        04 & Неоплачиваемые семейные работники \\
    \end{tabular}
\end{center}

\subsection{ОКИН: 91 --- Срок трудового договора (контракта)}

\begin{center}
    \begin{tabular}{rp{.8\textwidth}}
        \hline
        \textbf{Код} & \textbf{Наименование} \\ \hline
        01 & Неопределенный срок \\
        02 & Определенный срок не более 5 лет \\
    \end{tabular}
\end{center}

\subsection{ОКИН: 92 --- Дополнительная работа}

\begin{center}
    \begin{tabular}{rp{.8\textwidth}}
        \hline
        \textbf{Код} & \textbf{Наименование} \\ \hline
        01 & Совместительство, выполняемое/на постоянной основе \\
        02 & - на временной основе \\
        03 & - на сезонной основе \\
        04 & Дополнительная работа/по контракту \\
        05 & - случайная, разовая \\
        06 & - на индивидуальной основе \\
        07 & - в виде предпринимательской деятельности без образования юридического лица \\
        08 & - по найму у отдельных граждан \\
        09 & Работа, выполняемая сверх нормальной продолжительности рабочего времени по основному месту работы (дополнительная ставка)/Совмещение профессий (должностей) \\
    \end{tabular}
\end{center}

\subsection{ОКИН: 93 --- Работа в неблагоприятных условиях труда}

\begin{center}
    \begin{tabular}{rp{.8\textwidth}}
        \hline
        \textbf{Код} & \textbf{Наименование} \\ \hline
        01 & Работа с вредными и (или) опасными условиями труда \\
        02 & Работа под воздействием факторов производственной среды/шума, ультразвука, инфразвука \\
        03 & - вибрации (общей и локальной) \\
        04 & - аэрозолей преимущественно фиброгенного действия \\
        05 & - химического фактора \\
        06 & - неионизирующего излучения \\
        07 & - ионизирующего излучения \\
        08 & - нагревающего микроклимата \\
        09 & - охлаждающего микроклимата \\
        10 & - световой среды \\
        11 & - биологического фактора \\
        12 & Работа под воздействием факторов трудового процесса/тяжести \\
        13 & - напряженности \\
    \end{tabular}
\end{center}

\subsection{ОКИН: 102 — Случаи расторжения трудового договора по инициативе работодателя}

\begin{center}
    \begin{longtable}{rp{.8\textwidth}}
        \hline
        \textbf{Код} & \textbf{Наименование} \\ \hline
        10 & Ликвидация организации либо прекращение деятельности работодателем - физическим лицом \\
        20 & Сокращение численности или штата работников организации \\
        30 & Несоответствие работника занимаемой должности или выполняемой работе/ \\
        31 & - вследствие состояния здоровья в соответствии с медицинским заключением \\
        32 & - вследствие недостаточной квалификации, подтвержденной результатами аттестации \\
        40 & Смена собственника имущества организации (в отношении руководителя организации, его заместителей и главного бухгалтера) \\
        50 & Неоднократное неисполнение работником без уважительных причин трудовых обязанностей, если он имеет дисциплинарное взыскание \\
        60 & Однократное грубое нарушение работником трудовых обязанностей \\
        61 & Прогул (отсутствие на рабочем месте без уважительных причин более четырех часов подряд в течение рабочего дня) \\
        61 & Совершение по месту работы хищения (в том числе мелкого) чужого имущества, растраты, умышленного его уничтожения или повреждения, установленных вступившим в законную силу приговором суда или постановлением органа, уполномоченного на применение администрат \\
        62 & Появление на работе в состоянии алкогольного, наркотического или иного токсического опьянения \\
        63 & Разглашение охраняемой законом тайны (государственной, коммерческой, служебной и иной), ставшей известной работнику в связи с исполнением им трудовых обязанностей \\
        65 & Нарушение работником требований по охране труда, если это нарушение повлекло за собой тяжкие последствия (несчастный случай на производстве, авария, катастрофа) либо заведомо создавало реальную угрозу наступления таких последствий \\
        71 & Совершение виновных действий работником, непосредственно обслуживающим денежные или товарные ценности, если эти действия дают основание для утраты доверия к нему со стороны работодателя \\
        72 & Совершение работником, выполняющим воспитательные функции, аморального проступка, несовместимого с продолжением данной работы \\
        73 & Принятие необоснованного решения руководителем организации (филиала, представительства), его заместителями и главным бухгалтером, повлекшего за собой нарушение сохранности имущества, неправомерное его использование или иной ущерб имуществу организации \\
        74 & Однократное грубое нарушение руководителем организации (филиала, представительства), его заместителями своих трудовых обязанностей \\
        75 & Представление работником работодателю подложных документов или заведомо ложных сведений при заключении трудового договора \\
        76 & Прекращение допуска к государственной тайне, если выполняемая работа требует допуска к государственной тайне \\
        77 & Расторжение трудового договора по инициативе работодателя в случаях, предусмотренных трудовым договором с руководителем организации, членами коллегиального исполнительного органа организации \\
        78 & Другие случаи расторжения трудового договора работодателем, установленные законодательством Российской Федерации о труде \\
    \end{longtable}
\end{center}

\subsection{ОКИН: 103 --- Обстоятельства прекращения трудового договора, не зависящие от воли сторон}

\begin{center}
    \begin{tabular}{rp{.8\textwidth}}
        \hline
        \textbf{Код} & \textbf{Наименование} \\ \hline
        01 & Призыв работника на военную службу или направление его на заменяющую ее альтернативную гражданскую службу \\
        02 & Восстановление на  работе  работника,  ранее выполнявшего эту работу,  по решению государственной  инспекции  труда или суда Пояснение: Прекращение  трудового  договора  допускается, если  невозможно  перевести  работника  с его согласия на другую работу \\
        03 & Неизбрание на должность \\
        04 & Осуждение работника к наказанию, исключающему продолжение прежней   работы,   в  соответствии  с  приговором  суда, вступившим в законную силу \\
        05 & Признание работника    полностью    нетрудоспособным    в соответствии с медицинским заключением \\
        06 & Смерть работника либо работодателя - физического лица,  а также  признание  судом  работника  либо  работодателя  - физического лица умершим или безвестно отсутствующим \\
        07 & Наступление чрезвычайных   обстоятельств,  препятствующих продолжению   трудовых   отношений   (военные   действия, катастрофа,  стихийное бедствие, крупная авария, эпидемия и  другие  чрезвычайные  обстоятельства),   если   данное обстоятельство признано решением Правительства Российской Федерации    или    органа     государственной     власти соответствующего субъекта Российской Федерации \\
    \end{tabular}
\end{center}

\subsection{ОКИН: 104 --- Случаи прекращения трудового договора вследствие нарушения установленных законодательством Российской Федерации обязательных правил при заключении трудового договора}

\begin{center}
    \begin{tabular}{rp{.8\textwidth}}
        \hline
        \textbf{Код} & \textbf{Наименование} \\ \hline
        01 & Заключение трудового договора в нарушение приговора  суда о  лишении  конкретного  лица права занимать определенные должности или заниматься определенной деятельностью \\
        02 & Заключение трудового  договора  на   выполнение   работы, противопоказанной  данному  лицу  по состоянию здоровья в соответствии с медицинским заключением \\
        03 & Отсутствие соответствующего   документа  об  образовании, если  выполнение  работы  требует  специальных  знаний  в соответствии  с  законодательством Российской Федерации о труде или иным нормативным правовым актом \\
        04 & Другие случаи,     предусмотренные      законодательством Российской Федерации \\
    \end{tabular}
\end{center}

\subsection{ОКИН: 202 --- Места жительства населения в Российской Федерации}

\begin{center}
    \begin{tabular}{rp{.8\textwidth}}
        \hline
        \textbf{Код} & \textbf{Наименование} \\ \hline
        01 & Одноквартирный дом \\
        02 & Квартира \\
        03 & Комната в коммунальной квартире \\
        04 & Служебное жилое помещение \\
        10 & Специализированные дома \\
        11 & Общежитие \\
        12 & Гостиница-приют \\
        13 & Дом маневренного фонда \\
        14 & Специализированный дом для одиноких и престарелых \\
        15 & Дом-интернат для инвалидов, ветеранов \\
        16 & Детский дом \\
        17 & Интернат при школе \\
        18 & Школа-интернат специализированная \\
        19 & Специализированное учреждение для несовершеннолетних, нуждающихся в социальной реабилитации \\
        25 & Другие специализированные дома \\
        30 & Иное жилое помещение, в котором лицо постоянно или преимущественно проживает \\
        35 & Отсутствие постоянного места жительства \\
    \end{tabular}
\end{center}

\subsection{ОКИН: 203 --- Основания проживания населения}

\begin{center}
    \begin{tabular}{rp{.8\textwidth}}
        \hline
        \textbf{Код} & \textbf{Наименование} \\ \hline
        01 & Проживание лица/в качестве собственника \\
        02 & - по договору найма (поднайма) \\
        03 & - по договору социального найма \\
        04 & - по договору аренды \\
        09 & - на иных основаниях, предусмотренных законодательством Российской Федерации \\
    \end{tabular}
\end{center}





\section{Описание полей документа}

\subsection{Схема реляционного отношения}

\begin{center}
\begin{longtable}{ | m{.25\textwidth} | m{.01\textwidth}| m{.73\textwidth} | }
 
 \hline
 \firstColumn{Работник}{Изменял}{Старые ФИО}{\ruleOneOptionalManyMondatory} & \ruleOneOptionalManyMondatoryNum & \thirdColumn{\rabotnik}{\rabotnikPK}{\rabotnikFK}{\starieFIO}{\starieFIOPK}{\starieFIOFK} \\ 
 
 \hline
 \firstColumn{Работник}{Имеет}{Классификатор гражданства}{\ruleManyOptionalOneMondatory} & \ruleManyOptionalOneMondatoryNum & \thirdColumn{\rabotnik}{\rabotnikPK}{\rabotnikFK}{\kGrazhdanstva}{\kGrazhdanstvaPK}{\kGrazhdanstvaFK} \\ 
 
 \hline
 \firstColumn{Работник}{Имеет}{Профессиональная переподготовка}{\ruleOneOptionalManyMondatory} & \ruleOneOptionalManyMondatoryNum & \thirdColumn{\rabotnik}{\rabotnikPK}{\rabotnikFK}{\professionalnayaPerepodgatovka}{\professionalnayaPerepodgatovkaPK}{\professionalnayaPerepodgatovkaFK} \\ 
 
 \hline
 \firstColumn{Профессиональная переподготовка}{Определяет}{Классификатор должностей}{\ruleManyMondatoryOneOptional} & \ruleManyMondatoryOneOptionalNum & \thirdColumn{\professionalnayaPerepodgatovka}{\professionalnayaPerepodgatovkaPK}{\professionalnayaPerepodgatovkaFK}{\kDolzhostey}{\kDolzhosteyPK}{\kDolzhosteyFK} \\ 
 
 \hline
 \firstColumn{Классификатор должностей}{Имеется}{Штатные единицы}{\ruleOneOptionalManyMondatory} & \ruleOneOptionalManyMondatoryNum & \thirdColumn{\kDolzhostey}{\kDolzhosteyPK}{\kDolzhosteyFK}{\shtatnieEdinitsi}{\shtatnoyeRaspisanieOrganizatsiiPK}{\shtatnieEdinitsiPK} \\ 
 
 \hline
 \firstColumn{Штатные единицы}{Имеет}{Классификатор подразделения}{\ruleManyMondatoryOneMondatory} & \ruleManyMondatoryOneMondatoryNum & \thirdColumn{\shtatnieEdinitsi}{\shtatnieEdinitsiPK}{\shtatnieEdinitsiFK}{\kPodrazdeleniya}{\kPodrazdeleniyaPK}{\kPodrazdeleniyaFK} \\ 
 
 \hline
 \firstColumn{Штатные единицы}{Включает}{Штатное расписание организации}{\ruleManyMondatoryOneMondatory} & \ruleManyMondatoryOneMondatoryNum & \thirdColumn{\shtatnieEdinitsi}{\shtatnieEdinitsiPK}{\shtatnieEdinitsiFK}{\shtatnoyeRaspisanieOrganizatsii}{\shtatnoyeRaspisanieOrganizatsiiPK}{\shtatnoyeRaspisanieOrganizatsiiFK} \\ 
 
 \hline
 \firstColumn{Штатное расписание организации}{Утверждает}{Приказ}{\ruleOneMondatoryOneOptional} & \ruleOneMondatoryOneOptionalNum & \thirdColumn{\shtatnoyeRaspisanieOrganizatsii}{\shtatnoyeRaspisanieOrganizatsiiPK}{\shtatnoyeRaspisanieOrganizatsiiFK}{\prikaz}{\prikazPK}{\prikazFK} \\ 
 
 \hline
 \firstColumn{Приказ}{Соответствует}{Трудовой договор}{\ruleOneOptionalManyMondatory} & \ruleOneMondatoryManyMondatoryNum & \thirdColumn{\prikaz}{\prikazPK}{\prikazFK}{\trudovoiDogovor}{\trudovoiDogovorPK}{\trudovoiDogovorFK} \\ 
 
 \hline
 \firstColumn{Штатные единицы}{Принимается}{Трудовой договор}{\ruleOneMondatoryManyMondatory} & \ruleOneMondatoryManyMondatoryNum & \thirdColumn{\shtatnieEdinitsi}{\shtatnieEdinitsiPK}{\shtatnieEdinitsiFK}{\trudovoiDogovor}{\trudovoiDogovorPK}{\trudovoiDogovorFK} \\ 
 
 \hline
 \firstColumn{Трудовой договор}{Определяет}{Классификатор характера работы}{\ruleManyMondatoryOneOptional} & \ruleManyMondatoryOneOptionalNum & \thirdColumn{\trudovoiDogovor}{\trudovoiDogovorPK}{\trudovoiDogovorFK}{\kHarakteraRaboti}{\kHarakteraRabotiPK}{\kHarakteraRabotiFK} \\ 
 
 \hline
 \firstColumn{Трудовой договор}{Определяет}{Классификатор вида работы}{\ruleManyOptionalOneMondatory} & \ruleManyOptionalOneMondatoryNum & \thirdColumn{\trudovoiDogovor}{\trudovoiDogovorPK}{\trudovoiDogovorFK}{\kVidaRaboti}{\kDolzhosteyPK}{\attestatsiyaFK} \\ 
 
 \hline
 \firstColumn{Трудовой договор}{Определяет}{Классификатор основания прекращения трудовой деятельности}{\ruleManyOptionalOneMondatory} & \ruleManyOptionalOneMondatoryNum & \thirdColumn{\trudovoiDogovor}{\trudovoiDogovorPK}{\trudovoiDogovorFK}{\kOsnovaniyaPrekrascheniaTrudovogoDogovora}{\kOsnovaniyaPrekrascheniaTrudovogoDogovoraPK}{\kOsnovaniyaPrekrascheniaTrudovogoDogovoraFK} \\ 
 
 \hline
 \firstColumn{Работник}{Заключает}{Трудовой договор}{\ruleOneMondatoryManyMondatory} & \ruleOneMondatoryManyMondatoryNum & \thirdColumn{\rabotnik}{\rabotnikPK}{\rabotnikFK}{\trudovoiDogovor}{\trudovoiDogovorPK}{\trudovoiDogovorFK} \\ 

 \hline
 \firstColumn{Работник}{Имеет}{Страны}{\ruleManyOptionalOneOptional} & \ruleManyOptionalOneOptionalNum & \thirdColumn{\rabotnik}{\rabotnikPK}{\rabotnikFK}{\strani}{\straniPK}{\straniFK} \\ 
 
 \hline
 \firstColumn{Работник}{Аттестован}{Аттестация}{\ruleOneOptionalManyMondatory} & \ruleOneOptionalManyMondatoryNum & \thirdColumn{\rabotnik}{\rabotnikPK}{\rabotnikFK}{\attestatsiya}{\attestatsiyaPK}{\attestatsiyaFK} \\ 
 
 \hline
 \firstColumn{Аттестация}{Определяет}{Классификатор решения комиссии}{\ruleOneMondatoryOneOptional} & \ruleOneMondatoryOneOptionalNum & \thirdColumn{\attestatsiya}{\attestatsiyaPK}{\attestatsiyaFK}{\kResheniyKomissii}{\kResheniyKomissiiPK}{\kResheniyKomissiiFK} \\ 
 
 \hline
 \firstColumn{Работник}{Состоит}{Воинский учет}{\ruleOneOptionalOneMondatory} & \ruleOneOptionalOneMondatoryNum & \thirdColumn{\rabotnik}{\rabotnikPK}{\rabotnikFK}{\voinsiyUchet}{\voinsiyUchetPK}{\voinsiyUchetFK} \\ 
 
 \hline
 \firstColumn{Воинский учет}{Определяет}{Классификатор составов}{\ruleManyMondatoryOneOptional} & \ruleManyMondatoryOneOptionalNum & \thirdColumn{\voinsiyUchet}{\voinsiyUchetPK}{\voinsiyUchetFK}{\kSostavov}{\kSostavovPK}{\kSostavovFK} \\ 
 
 \hline
 \firstColumn{Воинский учет}{Определяет}{Классификатор воинского звания}{\ruleManyMondatoryOneOptional} & \ruleManyMondatoryOneOptionalNum & \thirdColumn{\voinsiyUchet}{\voinsiyUchetPK}{\voinsiyUchetFK}{\kVoinskihZvaniy}{\kVidaOtpuskaPK}{\kVidaOtpuskaFK} \\ 
 
 \hline
 \firstColumn{Работник}{Имеет}{Повышение квалификации}{\ruleOneOptionalManyMondatory} & \ruleOneOptionalManyMondatoryNum & \thirdColumn{\rabotnik}{\rabotnikPK}{\rabotnikFK}{\povishenieKvalifikatsii}{\povishenieKvalifikatsiiPK}{\povishenieKvalifikatsiiFK} \\ 
 
 \hline
 \firstColumn{Повышение квалификации}{Определяет}{Классификатор вида повышения квалификации}{\ruleOneMondatoryOneOptional} & \ruleOneMondatoryOneOptionalNum & \thirdColumn{\povishenieKvalifikatsii}{\povishenieKvalifikatsiiPK}{\povishenieKvalifikatsiiFK}{\kVidaPovisheniyaKvalifikatsii}{\kVidaPovisheniyaKvalifikatsiiPK}{\kVidaPovisheniyaKvalifikatsiiFK} \\ 
 
 \hline
 \firstColumn{Работник}{Знает}{Знание иностранного языка}{\ruleOneOptionalManyMondatory} & \ruleOneOptionalManyMondatoryNum & \thirdColumn{\rabotnik}{\rabotnikPK}{\rabotnikFK}{\znanieInostrannogoYazika}{\znanieInostrannogoYazikaPK}{\znanieInostrannogoYazikaFK} \\ 
 
 \hline
 \firstColumn{Знание иностранного языка}{Знают}{Классификатор иностранного языка}{\ruleManyMondatoryOneOptional} & \ruleManyMondatoryOneOptionalNum & \thirdColumn{\znanieInostrannogoYazika}{\znanieInostrannogoYazikaPK}{\znanieInostrannogoYazikaFK}{\kInostrannogoYazika}{\kInostrannogoYazikaPK}{\kInostrannogoYazikaFK} \\ 
 
 \hline
 \firstColumn{Знание иностранного языка}{Знает как}{Классификатор степени знания иностранного языка}{\ruleManyMondatoryOneOptional} & \ruleManyMondatoryOneOptionalNum & \thirdColumn{\znanieInostrannogoYazika}{\znanieInostrannogoYazikaPK}{\znanieInostrannogoYazikaFK}{\kStepeniZnaniaInostrannogoYazika}{\kStepeniZnaniaInostrannogoYazikaPK}{\kStepeniZnaniaInostrannogoYazikaFK} \\ 
 
 \hline
 \firstColumn{Работник}{Имеет}{Стаж работы}{\ruleOneMondatoryManyMondatory} & \ruleOneMondatoryManyMondatoryNum & \thirdColumn{\rabotnik}{\rabotnikPK}{\rabotnikFK}{\stazhRaboti}{\stazhRabotiPK}{\stazhRabotiFK} \\ 
 
 \hline
 \firstColumn{Работник}{Имеет}{Состояние в браке}{\ruleOneMondatoryManyMondatory} & \ruleOneMondatoryManyMondatoryNum & \thirdColumn{\rabotnik}{\rabotnikPK}{\rabotnikFK}{\sostoyaniyeVBrake}{\sostoyaniyeVBrakePK}{\sostoyaniyeVBrakeFK} \\ 
 
 \hline
 \firstColumn{Состояние в браке}{Определяет}{Классификатор состояния в браке}{\ruleManyMondatoryOneOptional} & \ruleManyMondatoryOneOptionalNum & \thirdColumn{\sostoyaniyeVBrake}{\shtatnieEdinitsiPK}{\shtatnieEdinitsiFK}{\kSostoyaniaVBrake}{\kSostoyaniaVBrakePK}{\kSostoyaniaVBrakeFK} \\ 
 
 \hline
 \firstColumn{Работник}{Имеет}{Родственники}{\ruleOneOptionalManyMondatory} & \ruleOneMondatoryManyMondatoryNum & \thirdColumn{\rabotnik}{\rabotnikPK}{\rabotnikFK}{\rodtvenniki}{\rodtvennikiPK}{\rodtvennikiFK} \\ 
 
 \hline
 \firstColumn{Родственники}{Определяет}{Классификатор родства}{\ruleManyMondatoryOneOptional} & \ruleManyMondatoryOneOptionalNum & \thirdColumn{\rodtvenniki}{\rodtvennikiPK}{\rodtvennikiFK}{\kRodstva}{\kRodstvaPK}{\kRodstvaFK} \\ 
 
 \hline
 \firstColumn{Работник}{Имеет}{Документ об образовании}{\ruleOneOptionalManyMondatory} & \ruleOneOptionalManyMondatoryNum & \thirdColumn{\rabotnik}{\rabotnikPK}{\rabotnikFK}{\documentObObrazovanii}{\documentObObrazovaniiPK}{\documentObObrazovaniiFK} \\ 
 
 \hline
 \firstColumn{Документ об образовании}{Подтверждает}{Классификатор направления}{\ruleManyOptionalOneOptional} & \ruleManyOptionalOneOptionalNum & \thirdColumn{\documentObObrazovanii}{\documentObObrazovaniiPK}{\documentObObrazovaniiFK}{\kNapravlenia}{\kNagradPK}{\kNapravleniaFK} \\ 
 
 \hline
 \firstColumn{Работник}{Имеет}{Классификатор образования}{\ruleManyMondatoryOneOptional} & \ruleManyMondatoryOneOptionalNum & \thirdColumn{\rabotnik}{\rabotnikPK}{\rabotnikFK}{\kObrazovania}{\kDolzhosteyPK}{\kObrazovaniaFK} \\ 
 
 \hline
 \firstColumn{Работник}{Имеет}{Социальные льготы}{\ruleOneOptionalManyMondatory} & \ruleOneOptionalManyMondatoryNum & \thirdColumn{\rabotnik}{\rabotnikPK}{\rabotnikFK}{\sotsialnieLgoti}{\sostoyaniyeVBrakePK}{\sotsialnieLgotiFK} \\ 
 
 \hline
 \firstColumn{Работник}{Имеет}{Отпуск}{\ruleOneOptionalManyMondatory} & \ruleOneOptionalManyMondatoryNum & \thirdColumn{\rabotnik}{\rabotnikPK}{\rabotnikFK}{\otpusk}{\otpuskPK}{\otpuskFK} \\ 
 
 \hline
 \firstColumn{Отпуск}{Определяет}{Классификатор вида отпуска}{\ruleManyMondatoryOneOptional} & \ruleManyMondatoryOneOptionalNum & \thirdColumn{\otpusk}{\otpuskPK}{\otpuskFK}{\kVidaOtpuska}{\kVidaOtpuskaPK}{\kVidaOtpuskaFK} \\ 
 
 \hline
 \firstColumn{Работник}{Награжден}{Награды, почетные звания}{\ruleOneOptionalManyMondatory} & \ruleOneMondatoryManyMondatoryNum & \thirdColumn{\rabotnik}{\rabotnikPK}{\rabotnikFK}{\nagradiPochetnieZvaniya}{\nagradiPochetnieZvaniyaPK}{\nagradiPochetnieZvaniyaFK} \\ 
 
 \hline
 \firstColumn{Награды, почетные звания}{Определяет}{Классификатор наград}{\ruleManyMondatoryOneOptional} & \ruleManyMondatoryOneOptionalNum & \thirdColumn{\nagradiPochetnieZvaniya}{\nagradiPochetnieZvaniyaPK}{\nagradiPochetnieZvaniyaFK}{\kNagrad}{\kNagradPK}{\kNagradFK} \\ 
 
 %\hline
 %\firstColumn{}{}{}{\rule} & \ruleNum & \thirdColumn{}{}{}{}{}{} \\ 
 
 \hline
\end{longtable}
\end{center}

%\pagebreak
\subsection{Обобщенная схема реляционного отношения}

\begin{center}
\begin{longtable}{ | m{.3\textwidth} | m{.7\textwidth} | } 

 \hline
 \firstColumn{Работник}{Изменял}{Старые ФИО}{\ruleOneOptionalManyMondatory} & \generalizedColumn{Работник}{\rabotnikPK}{\rabotnikFK}{Старые ФИО}{\starieFIOPK}{\starieFIOFK} \\ 
 
 \hline
 \firstColumn{Работник}{Имеет}{Классификатор гражданства}{\ruleManyOptionalOneMondatory} & \generalizedColumn{Работник}{\rabotnikPK}{\rabotnikFK}{Классификатор гражданства}{\kGrazhdanstvaPK}{\kGrazhdanstvaFK} \\ 
 
 \hline
 \firstColumn{Работник}{Имеет}{Профессиональная переподготовка}{\ruleOneOptionalManyMondatory} & \generalizedColumn{Работник}{\rabotnikPK}{\rabotnikFK}{Профессиональная переподготовка}{\professionalnayaPerepodgatovkaPK}{\professionalnayaPerepodgatovkaFK} \\ 
 
 \hline
 \firstColumn{Профессиональная переподготовка}{Определяет}{Классификатор должностей}{\ruleManyMondatoryOneOptional}  & \generalizedColumn{Профессиональная переподготовка}{\professionalnayaPerepodgatovkaPK}{\professionalnayaPerepodgatovkaFK}{Классификатор должностей}{\kDolzhosteyPK}{\kDolzhosteyFK} \\ 
 
 \hline
 \firstColumn{Классификатор должностей}{Имеется}{Штатные единицы}{\ruleOneOptionalManyMondatory} & \generalizedColumn{Классификатор должностей}{\kDolzhosteyPK}{\kDolzhosteyFK}{Штатные единицы}{\shtatnoyeRaspisanieOrganizatsiiPK}{\shtatnieEdinitsiPK} \\ 
 
 \hline
 \firstColumn{Штатные единицы}{Имеет}{Классификатор подразделения}{\ruleManyMondatoryOneMondatory} & \generalizedColumn{Штатные единицы}{\shtatnieEdinitsiPK}{\shtatnieEdinitsiFK}{Классификатор подразделения}{\kPodrazdeleniyaPK}{\kPodrazdeleniyaFK} \\ 
 
 \hline
 \firstColumn{Штатные единицы}{Включает}{Штатное расписание организации}{\ruleManyMondatoryOneMondatory} & \generalizedColumn{Штатные единицы}{\shtatnieEdinitsiPK}{\shtatnieEdinitsiFK}{Штатное расписание организации}{\shtatnoyeRaspisanieOrganizatsiiPK}{\shtatnoyeRaspisanieOrganizatsiiFK} \\ 
 
 \hline
 \firstColumn{Штатное расписание организации}{Утверждает}{Приказ}{\ruleOneMondatoryOneOptional} & \generalizedColumn{Штатное расписание организации}{\shtatnoyeRaspisanieOrganizatsiiPK}{\shtatnoyeRaspisanieOrganizatsiiFK}{Приказ}{\prikazPK}{\prikazFK} \\ 
 
 \hline
 \firstColumn{Приказ}{Соответствует}{Трудовой договор}{\ruleOneOptionalManyMondatory} & \generalizedColumn{Приказ}{\prikazPK}{\prikazFK}{Трудовой договор}{\trudovoiDogovorPK}{\trudovoiDogovorFK} \\ 
 
 \hline
 \firstColumn{Штатные единицы}{Принимается}{Трудовой договор}{\ruleOneMondatoryManyMondatory} & \generalizedColumn{Штатные единицы}{\shtatnieEdinitsiPK}{\shtatnieEdinitsiFK}{Трудовой договор}{\trudovoiDogovorPK}{\trudovoiDogovorFK} \\ 
 
 \hline
 \firstColumn{Трудовой договор}{Определяет}{Классификатор характера работы}{\ruleManyMondatoryOneOptional} & \generalizedColumn{Трудовой договор}{\trudovoiDogovorPK}{\trudovoiDogovorFK}{Классификатор характера работы}{\kHarakteraRabotiPK}{\kHarakteraRabotiFK} \\ 
 
 \hline
 \firstColumn{Трудовой договор}{Определяет}{Классификатор вида работы}{\ruleManyOptionalOneMondatory} & \generalizedColumn{Трудовой договор}{\trudovoiDogovorPK}{\trudovoiDogovorFK}{Классификатор вида работы}{\kDolzhosteyPK}{\attestatsiyaFK} \\ 
 
 \hline
 \firstColumn{Трудовой договор}{Определяет}{Классификатор основания прекращения трудовой деятельности}{\ruleManyOptionalOneMondatory} & \generalizedColumn{Трудовой договор}{\trudovoiDogovorPK}{\trudovoiDogovorFK}{Классификатор основания прекращения трудовой деятельности}{\kOsnovaniyaPrekrascheniaTrudovogoDogovoraPK}{\kOsnovaniyaPrekrascheniaTrudovogoDogovoraFK} \\ 
 
 \hline
 \firstColumn{Работник}{Заключает}{Трудовой договор}{\ruleOneMondatoryManyMondatory} & \generalizedColumn{Работник}{\rabotnikPK}{\rabotnikFK}{Трудовой договор}{\trudovoiDogovorPK}{\trudovoiDogovorFK} \\ 

 \hline
 \firstColumn{Работник}{Имеет}{Страны}{\ruleManyOptionalOneOptional} & \generalizedColumn{Работник}{\rabotnikPK}{\rabotnikFK}{Страны}{\straniPK}{\straniFK} \\ 
 
 \hline
 \firstColumn{Работник}{Аттестован}{Аттестация}{\ruleOneOptionalManyMondatory} & \generalizedColumn{Работник}{\rabotnikPK}{\rabotnikFK}{Аттестация}{\attestatsiyaPK}{\attestatsiyaFK} \\ 
 
 \hline
 \firstColumn{Аттестация}{Определяет}{Классификатор решения комиссии}{\ruleOneMondatoryOneOptional} & \generalizedColumn{Аттестация}{\attestatsiyaPK}{\attestatsiyaFK}{Классификатор решения комиссии}{\kResheniyKomissiiPK}{\kResheniyKomissiiFK} \\ 
 
 \hline
 \firstColumn{Работник}{Состоит}{Воинский учет}{\ruleOneOptionalOneMondatory} & \generalizedColumn{Работник}{\rabotnikPK}{\rabotnikFK}{Воинский учет}{\voinsiyUchetPK}{\voinsiyUchetFK} \\ 
 
 \hline
 \firstColumn{Воиский учет}{Определяет}{Классификатор составов}{\ruleManyMondatoryOneOptional} & \generalizedColumn{Воиский учет}{\voinsiyUchetPK}{\voinsiyUchetFK}{Классификатор составов}{\kSostavovPK}{\kSostavovFK} \\ 
 
 \hline
 \firstColumn{Воинский учет}{Определяет}{Классификатор воинского звания}{\ruleManyMondatoryOneOptional} & \generalizedColumn{Воинский учет}{\voinsiyUchetPK}{\voinsiyUchetFK}{Классификатор воинского звания}{\kVidaOtpuskaPK}{\kVidaOtpuskaFK} \\ 
 
 \hline
 \firstColumn{Работник}{Имеет}{Повышение квалификации}{\ruleOneOptionalManyMondatory} & \generalizedColumn{Работник}{\rabotnikPK}{\rabotnikFK}{Повышение квалификации}{\povishenieKvalifikatsiiPK}{\povishenieKvalifikatsiiFK} \\ 
 
 \hline
 \firstColumn{Повышение квалификации}{Определяет}{Классификатор вида повышения квалификации}{\ruleOneMondatoryOneOptional} & \generalizedColumn{Повышение квалификации}{\povishenieKvalifikatsiiPK}{\povishenieKvalifikatsiiFK}{Классификатор вида повышения квалификации}{\kVidaPovisheniyaKvalifikatsiiPK}{\kVidaPovisheniyaKvalifikatsiiFK} \\ 
 
 \hline
 \firstColumn{Работник}{Знает}{Знание иностранного языка}{\ruleOneOptionalManyMondatory} & \generalizedColumn{Работник}{\rabotnikPK}{\rabotnikFK}{Знание иностранного языка}{\znanieInostrannogoYazikaPK}{\znanieInostrannogoYazikaFK} \\ 
 
 \hline
 \firstColumn{Знание иностранного языка}{Знают}{Классификатор иностранного языка}{\ruleManyMondatoryOneOptional} & \generalizedColumn{Знание иностранного языка}{\znanieInostrannogoYazikaPK}{\znanieInostrannogoYazikaFK}{Классификатор иностранного языка}{\kInostrannogoYazikaPK}{\kInostrannogoYazikaFK} \\ 
 
 \hline
 \firstColumn{Знание иностранного языка}{Знает как}{Классификатор степени знания иностранного языка}{\ruleManyMondatoryOneOptional} & \generalizedColumn{Знание иностранного языка}{\znanieInostrannogoYazikaPK}{\znanieInostrannogoYazikaFK}{Классификатор степени знания иностранного языка}{\kStepeniZnaniaInostrannogoYazikaPK}{\kStepeniZnaniaInostrannogoYazikaFK} \\ 
 
 \hline
 \firstColumn{Работник}{Имеет}{Стаж работы}{\ruleOneMondatoryManyMondatory} & \generalizedColumn{Работник}{\rabotnikPK}{\rabotnikFK}{Стаж работы}{\stazhRabotiPK}{\stazhRabotiFK} \\ 
 
 \hline
 \firstColumn{Работник}{Имеет}{Состояние в браке}{\ruleOneMondatoryManyMondatory} & \generalizedColumn{Работник}{\rabotnikPK}{\rabotnikFK}{Состояние в браке}{\sostoyaniyeVBrakePK}{\sostoyaniyeVBrakeFK} \\ 
 
 \hline
 \firstColumn{Состояние в браке}{Определяет}{Классификатор состояния в браке}{\ruleManyMondatoryOneOptional} & \generalizedColumn{Состояние в браке}{\shtatnieEdinitsiPK}{\shtatnieEdinitsiFK}{Классификатор состояния в браке}{\kSostoyaniaVBrakePK}{\kSostoyaniaVBrakeFK} \\ 
 
 \hline
 \firstColumn{Работник}{Имеет}{Родственники}{\ruleOneOptionalManyMondatory} & \generalizedColumn{Работник}{\rabotnikPK}{\rabotnikFK}{Родственники}{\rodtvennikiPK}{\rodtvennikiFK} \\ 
 
 \hline
 \firstColumn{Родственники}{Определяет}{Классификатор родства}{\ruleManyMondatoryOneOptional} & \generalizedColumn{Родственники}{\rodtvennikiPK}{\rodtvennikiFK}{Классификатор родства}{\kRodstvaPK}{\kRodstvaFK} \\ 
 
 \hline
 \firstColumn{Работник}{Имеет}{Документ об образовании}{\ruleOneOptionalManyMondatory} & \generalizedColumn{Работник}{\rabotnikPK}{\rabotnikFK}{Документ об образовании}{\documentObObrazovaniiPK}{\documentObObrazovaniiFK} \\ 
 
 \hline
 \firstColumn{Документ об образовании}{Подтверждает}{Классификатор направления}{\ruleManyOptionalOneOptional} & \generalizedColumn{Документ об образовании}{\documentObObrazovaniiPK}{\documentObObrazovaniiFK}{Классификатор направления}{\kNagradPK}{\kNapravleniaFK} \\ 
 
 \hline
 \firstColumn{Работник}{Имеет}{Классификатор образования}{\ruleManyMondatoryOneOptional} & \generalizedColumn{Работник}{\rabotnikPK}{\rabotnikFK}{Классификатор образования}{\kDolzhosteyPK}{\kObrazovaniaFK} \\ 
 
 \hline
 \firstColumn{Работник}{Имеет}{Социальные льготы}{\ruleOneOptionalManyMondatory} & \generalizedColumn{Работник}{\rabotnikPK}{\rabotnikFK}{Социальные льготы}{\sostoyaniyeVBrakePK}{\sotsialnieLgotiFK} \\ 
 
 \hline
 \firstColumn{Работник}{Имеет}{Отпуск}{\ruleOneOptionalManyMondatory} & \generalizedColumn{Работник}{\rabotnikPK}{\rabotnikFK}{Отпуск}{\otpuskPK}{\otpuskFK} \\ 
 
 \hline
 \firstColumn{Отпуск}{Определяет}{Классификатор вида отпуска}{\ruleManyMondatoryOneOptional} & \generalizedColumn{Отпуск}{\otpuskPK}{\otpuskFK}{Классификатор вида отпуска}{\kVidaOtpuskaPK}{\kVidaOtpuskaFK} \\ 
 
 \hline
 \firstColumn{Работник}{Награжден}{Награды, почетные звания}{\ruleOneOptionalManyMondatory} & \generalizedColumn{Работник}{\rabotnikPK}{\rabotnikFK}{Награды, почетные звания}{\nagradiPochetnieZvaniyaPK}{\nagradiPochetnieZvaniyaFK} \\ 
 
 \hline
 \firstColumn{Награды, почетные звания}{Определяет}{Классификатор наград}{\ruleManyMondatoryOneOptional} & \generalizedColumn{Награды, почетные звания}{\nagradiPochetnieZvaniyaPK}{\nagradiPochetnieZvaniyaFK}{Классификатор наград}{\kNagradPK}{\kNagradFK} \\ 
 
 %\hline
 %\firstColumn{}{}{}{\rule} & \generalizedColumn{}{}{}{}{}{} \\ 
 
 \hline
\end{longtable}
\end{center}

\end{document}
